\subsection{Class Changes}
\subsubsection{Modified Classes}
\paragraph*{\textcolor{Blue}{Class: Blabla}}
\paragraph*{}
\begin{longtable}{llp{10cm}}
	\hline\rowcolor{white}\textbf{Method Before} & \textbf{Method After} & \textbf{Note} \\ \hline
	\rowcolor{gray!20}method before & name after & sounds nicer \\
	\rowcolor{white}name before & name after & looks nicer \\
	\rowcolor{gray!20}name before & name after & looks nicer \\
	\hline
\end{longtable}

\pagebreak

\subsubsection{Added Classes}
% GraphioliLogFormatter
\class{GraphioliLogFormatter}{graphiolilogformatter}
\createindentedlist{java.lang.Object, java.util.logging.Formatter, de.graphioli.utils.GraphioliLogFormatter}
This class is used by the \ref{cls:graphiolilogger} to generate HTML formatted log messages. \\

\centerdash

\paragraph*{Method Summary}
\paragraph*{}
\begin{longtable}{Lp{10cm}}
	\startmethodtable
	\method{private static String}{generateTimeStamp()}{glf:generatetimestamp} \\
	& Generates a formatted date string of the time this method was called. \\
	\method{public String}{format(LogRecord rec)}{glf:format} \\
	& Format the given log record and return the formatted string. \\
	\method{public String}{getHead(Handler h)}{glf:gethead} \\
	& Return the header string for a set of formatted records. \\
	\method{public String}{getTail(Handler h)}{glf:gettail} \\
	& Return the tail string for a set of formatted records. \\
	\hline
\end{longtable}

\pagebreak

% GraphioliLogger
\class{GraphioliLogger}{graphiolilogger}
This class is used to create a HTML formatted log file for the Graphioli framework.

\centerdash

\paragraph*{Method Summary}
\paragraph*{}
\begin{longtable}{Lp{10cm}}
	\startmethodtable
	\method{public static void}{startLog(Level logLevel)}{gl:startlog} \\
	& Creates the log file and registers this logger. \\
	\hline
\end{longtable}

% InvalidJarException
\class{InvalidJarException}{invalidjarexception}
\createindentedlist{java.lang.Object, java.lang.Throwable, java.lang.Exception, de.graphioli.utils.InvalidJarException}
This exception gets thrown, when a Jar file does not match the format for a game.

\centerdash

\paragraph*{Method Summary}
\paragraph*{}
\begin{longtable}{Lp{10cm}}
	\startmethodtable
	\method{public}{InvalidJarException()}{ije:invalidjarexception} \\
	& Creates a new InvalidJarException. \\
	\method{public}{InvalidJarException(String msg)}{ije:invalidjarexceptionoverload} \\
	& Creates a new InvalidJarException with a message. \\
	\hline
\end{longtable}

\pagebreak

% JarParser
\static{JarParser}{jarparser}
This class is part of the \texttt{de.graphioli.utils} package and is able to scan a jar file for its content and return those, if neccessary. \\

\paragraph*{Method Summary}
\paragraph*{}
\begin{longtable}{Lp{10cm}}
	\startmethodtable
	\method{public static InputStream}{getFileAsInputStream(String gameName, String fileName)}{jp:getfileasinputstream} \\
	& Returns a file of a given jar as InputStream. \\
	\method{public static Reader}{getPropertyFile(String gameName)}{jp:getpropertyfile} \\
	& Returns the property file of a given game as \texttt{Reader}. \\
	\method{public static Class}{getClass(String gamePackagePath, String gameName)}{jp:getclass} \\
	& Returns the game's class file within the jar. \\
	\method{public static URI}{getHelpFileURI(String gameName)}{jp:gethelpfileuri} \\
	& Gets the help file within the jar, parses it to a string and convert it to a URI. \\
	\hline
\end{longtable}

% Localization
\static{Localization}{localization}
This class is responsible for loading the correct localization file.

\paragraph*{Method Summary}
\paragraph*{}
\begin{longtable}{Lp{10cm}}
	\startmethodtable
	\method{public static String}{getLanguageString(String key)}{locale:getlanguagestring} \\
	& Gets the respective string to a given key value.
\end{longtable}