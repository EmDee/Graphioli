\subsubsection{Added Classes}

% InvalidJarException
\class{InvalidJarException}{invalidjarexception}
\createindentedlist{java.lang.Object, java.lang.Throwable, java.lang.Exception, de.graphioli.utils.InvalidJarException}
This exception gets thrown, when a Jar file does not match the format for a game.

\centerdash

\paragraph*{Method Summary}
\paragraph*{}
\begin{longtable}{Lp{10cm}}
	\startmethodtable
	\method{public}{InvalidJarException()}{ije:invalidjarexception} \\
	& Creates a new InvalidJarException. \\
	\method{public}{InvalidJarException(String msg)}{ije:invalidjarexceptionoverload} \\
	& Creates a new InvalidJarException with a message. \\
	\hline
\end{longtable}

\pagebreak

% JarParser
\static{JarParser}{jarparser}
This class is part of the \texttt{de.graphioli.utils} package and is able to scan a jar file for its content and return those, if neccessary. \\

\paragraph*{Method Summary}
\paragraph*{}
\begin{longtable}{Lp{10cm}}
	\startmethodtable
	\method{public static InputStream}{getFileAsInputStream(String gameName, String fileName)}{jp:getfileasinputstream} \\
	& Returns a file of a given jar as InputStream. \\
	\method{public static Reader}{getPropertyFile(String gameName)}{jp:getpropertyfile} \\
	& Returns the property file of a given game as \texttt{Reader}. \\
	\method{public static Class}{getClass(String gamePackagePath, String gameName, ClassLoader parentClassLoader)}{jp:getclass} \\
	& Returns the game's class file within the jar. \\
	\method{public static URI}{getHelpFileURI(String gameName)}{jp:gethelpfileuri} \\
	& Gets the help file within the jar, parses it to a string and convert it to a URI. \\
	\hline
\end{longtable}

% GraphioliLogger
\class{GraphioliLogger}{graphiolilogger}
This class is used to create a HTML formatted log file for the Graphioli framework.

\centerdash

\paragraph*{Method Summary}
\paragraph*{}
\begin{longtable}{Lp{10cm}}
	\startmethodtable
	\method{public static void}{startLog(Level logLevel)}{gl:startlog} \\
	& Creates the log file and registers this logger. \\
	\hline
\end{longtable}

% GraphioliLogFormatter
\class{GraphioliLogFormatter}{graphiolilogformatter}
\createindentedlist{java.lang.Object, java.util.logging.Formatter, de.graphioli.utils.GraphioliLogFormatter}
This class is used by the \ref{cls:graphiolilogger} to generate HTML formatted log messages. \\

\centerdash

\paragraph*{Method Summary}
\paragraph*{}
\begin{longtable}{Lp{10cm}}
	\startmethodtable
	\method{private static String}{generateTimeStamp()}{glf:generatetimestamp} \\
	& Generates a formatted date string of the time this method was called. \\
	\method{public String}{format(LogRecord rec)}{glf:format} \\
	& Format the given log record and return the formatted string. \\
	\method{public String}{getHead(Handler h)}{glf:gethead} \\
	& Return the header string for a set of formatted records. \\
	\method{public String}{getTail(Handler h)}{glf:gettail} \\
	& Return the tail string for a set of formatted records. \\
	\hline
\end{longtable}

\pagebreak

% GEWindowActions
\class{GEWindowActions}{gewindowactions}
This class handles the action and event listeners for the GEWindow.

\paragraph*{Method Summary}
\paragraph*{}
\begin{longtable}{Lp{10cm}}
	\startmethodtable
	\method{public}{GEWindowActions(GEWindow geWindow)}{gewa:gewindowactions} \\
	& Constructs a new GEWindowActions action listener. \\
	\method{public void}{keyPressed(KeyEvent e)}{gewa:keypresses} \\
	& Listens to keyboard key presses. \\
	\method{public void}{mouseClicked(MouseEvent e)}{gewa:mouseclicked} \\
	& Listens to mouse clicks. \\
	\method{public void}{valueChanged(ListSelectionEvent event)}{gewa:valuechanged} \\
	& Listens to the list of available game definitions. \\
	\hline
\end{longtable}

% Localization
\static{Localization}{localization}
This class is responsible for loading the correct localization file.

\paragraph*{Method Summary}
\paragraph*{}
\begin{longtable}{Lp{10cm}}
	\startmethodtable
	\method{public static String}{getLanguageString(String key)}{locale:getlanguagestring} \\
	& Gets the respective string to a given key value. \\
	\hline
\end{longtable}

\pagebreak

% OptionsMenuItem
\class{OptionsMenuItem}{optionsmenuitem}
\createindentedlist{java.lang.Object, java.awt.Component, java.awt.Container, javax.swing.JComponent, javax.swing.AbstractButton, javax.swing.JMenuItem, de.graphioli.view.OptionsMenuItem}
This class implements the \texttt{MenuItem} as a Swing \texttt{JMenuItem}.

\paragraph*{Method Summary}
\paragraph*{}
\begin{longtable}{Lp{10cm}}
	\startmethodtable
	\method{public}{OptionsMenuItem(MenuItem item)}{omi:optionsmenuitem} \\
	& Creates a OptionsMenuItem from the given MenuItem. \\
	\method{protected MenuItem}{getCustomItem()}{omi:getcustomitem} \\
	& Returns the menu item. \\
	\hline
\end{longtable}

\pagebreak

% ClassLoaderObjectInputStream
\class{ClassLoaderObjectInputStream}{classloaderobjectinputstream}
\createindentedlist{java.lang.Object, java.io.InputStream, java.io.ObjectInputStream, de.graphioli.utils.ClassLoaderObjectInputStream}
This class is used for deserializing objects that need to be loaded by a specific class loader.

\paragraph*{Method Summary}
\paragraph*{}
\begin{longtable}{Lp{10cm}}
	\startmethodtable
	\method{public}{ClassLoaderObjectInputStream(InputStream in, ClassLoader classLoader)}{clois:clois} \\
	& Creates a \texttt{ClassLoaderObjectInputStream} based on the given input stream and class loader.. \\
	\method{public Class}{resolveClass(ObjectStreamClass desc)}{clois:resolveclass} \\
	& Load the local class equivalent of the specified stream class description. \\
	\hline
\end{longtable}

