\section{Introduction}
Game premised on graph algorithm have always been played as a pastime. Mostly without the knowledge about the underlying mathematical theories, which are -- of course -- not always relevant for clearing level after level. While two humans playing a graph game usually agree on the eventuation of one player winning the game, the same game on a computer must be computed to declare one player the winner.\par

This is where \graphioli takes effect. The Java framework simplifies the process of developing graph-based computer games by providing a straightforward, intuitive library that allows its user to focus the effort to the game's logical implementation. Recurring and redundant operations, algorithms and codings will be taken care of by \graphioli.\par

The framework still does not release the developer from actual programming work. However, it reduces the amount of time formerly spent realizing graphical outputs, structures and user interactions, thus reinventing the wheel.\par

Products based on \graphioli will be well-structured, lightweight in terms of code size and easy to maintain.\par

\subsection{About this document}
This document developed during validation and quality assurance phase and is divided into three parts:\par
In Section \ref{section:tests} we introduce the four different types of tests conducted: Unit tests, algorithm regressions tests, GUI tests and the manual test cases and scenarios.\par
Section \ref{section:results} describes the concrete tests and outlines their outcomes in tables and lists. Statistics about results and code coverage are provided.\par
The last section, Section \ref{section:corrected-errors}, documents the errors that were corrected during the phase.\par