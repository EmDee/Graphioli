\section{Results}
\subsection{Unit tests}

\todo{Results and statistics of unit tests here.}

\subsection{GUI tests}

\todo{Results of SIKULI tests here.}

\subsection{Manual test cases}

\subsubsection{Game-Explorer}

\begin{tabular}{clll}
	\hline
	\textbf{Test} & \textbf{Description} & \textbf{Result} & \textbf{Remark} \\
	\hline
	\ref{T:010} & \ref{T:010T} & Passed & \\
	\ref{T:020} & \ref{T:020T} & Passed & \\
	\ref{T:030} & \ref{T:030T} & Passed & \\
	\ref{T:040} & \ref{T:040T} & Failed & Implementation changed \\
	\hline
\end{tabular}

\begin{description}
	\item[\textlabel{/T010/}{T:010}] \textbf{\textlabel{Start the Game-Explorer}{T:010T}} \\
	\textbf{Input:} The tester clicks the Game-Explorer icon. \\
	\textbf{Exp. Output:} The Game-Explorer window opens, the game folder is scanned and all games contained are shown in the games list.
	
	\item[\textlabel{/T020/}{T:020}] \textbf{\textlabel{Select a game}{T:020T}} \\
	\textbf{Input:} The tester clicks on the name of the game in the list. \\
	\textbf{Exp. Output:} An image and a description of the game are displayed.
	
	\item[\textlabel{/T030/}{T:030}] \textbf{\textlabel{Start a game}{T:030T}} \\
	\textbf{Input:} A game has been selected and the tester clicks on the start button. \\
	\textbf{Exp. Output:} The game window opens.
	
	\item[\textlabel{/T040/}{T:040}] \textbf{\textlabel{Use the help function}{T:040T}} \\
	\textbf{Input:} A game has been selected and the tester clicks on the help button. \\
	\textbf{Exp. Output:} The help page is displayed in a new window.
\end{description}

\subsubsection\graphcoloring

\begin{tabular}{clll}
	\hline
	\textbf{Test} & \textbf{Description} & \textbf{Result} & \textbf{Remark} \\
	\hline
	\ref{T:050} & \ref{T:050T} & Not verified & \\
	\ref{T:060} & \ref{T:060T} & Not verified & \\
	\ref{T:070} & \ref{T:070T} & Not verified & \\
	\ref{T:080} & \ref{T:080T} & Not verified & \\
	\ref{T:090} & \ref{T:090T} & Not verified & \\
	\ref{T:100} & \ref{T:100T} & Not verified & \\
	\ref{T:110} & \ref{T:110T} & Not verified & \\
	\hline
\end{tabular}

\begin{description}
	\item[\textlabel{/T050/}{T:050}] \textbf{\textlabel{Start the game}{T:050T}} \\
	\textbf{Input:} \graphcoloring has been selected and the tester clicks on the start button. \\
	\textbf{Exp. Output:} The game window opens and the graph of the first level is displayed.
	
	\item[\textlabel{/T060/}{T:060}] \textbf{\textlabel{Save and load a game}{T:060T}} \\
	\textbf{Input:} Some vertices have been colored. The tester saves the current state, closes and reopens \graphcoloring and loads the savegame. \\
	\textbf{Exp. Output:} The state is saved in a file and when the file is loaded, the saved state is restored.
	
	\item[\textlabel{/T070/}{T:070}] \textbf{\textlabel{Color an uncolored vertex}{T:070T}} \\
	\textbf{Input:} The tester selects a color and clicks on an uncolored vertex that is not adjacent to one in the same color. \\
	\textbf{Exp. Output:} The vertex changes its color to the selected one.
	
	\item[\textlabel{/T080/}{T:080}] \textbf{\textlabel{Color a colored vertex}{T:080T}} \\
	\textbf{Input:} The tester selects a color and clicks on a colored vertex. \\
	\textbf{Exp. Output:} Nothing changes.
	
	\item[\textlabel{/T090/}{T:090}] \textbf{\textlabel{Color adjacent vertices in the same color}{T:090T}} \\
	\textbf{Input:} The tester selects a color and clicks on an uncolored vertex that is adjacent to one in the same color. \\
	\textbf{Exp. Output:} The vertex does not change. An error message is displayed.
	
	\item[\textlabel{/T100/}{T:100}] \textbf{\textlabel{Win/Lose a single-player game}{T:100T}} \\
	\textbf{Input:} The tester plays until he or she wins/loses the game. \\
	\textbf{Exp. Output:} A win/lose message is displayed and the next/same level is loaded.

	\item[\textlabel{/T110/}{T:110}] \textbf{\textlabel{Win/Lose a multiplayer game}{T:110T}} \\
	\textbf{Input:} The tester plays for both players until one wins the game. \\
	\textbf{Exp. Output:} A win message for the winning player is displayed.
	
\end{description}

\subsubsection\twixt

\begin{tabular}{clll}

\hline
	\textbf{Test} & \textbf{Description} & \textbf{Result} & \textbf{Remark} \\
	\hline
	\ref{T:120} & \ref{T:120T} & Passed & \\
	\ref{T:130} & \ref{T:130T} & Passed & \\
	\ref{T:140} & \ref{T:140T} & Passed & \\
	\ref{T:150} & \ref{T:150T} & Passed & \\
	\ref{T:160} & \ref{T:160T} & Passed & \\
	\ref{T:170} & \ref{T:170T} & Passed & \\
	\ref{T:180} & \ref{T:180T} & Passed & \\
	\hline
\end{tabular}

\begin{description}
	\item[\textlabel{/T120/}{T:120}] \textbf{\textlabel{Start the game}{T:120T}} \\
	\textbf{Input:} \twixt has been selected and the tester clicks on the start button. \\
	\textbf{Exp. Output:} The game window opens and the empty grid is displayed.
	
	\item[\textlabel{/T130/}{T:130}] \textbf{\textlabel{Save and load a game}{T:130T}} \\
	\textbf{Input:} Some vertices have been placed. The tester saves the current state, closes and reopens \twixt and loads the savegame. \\
	\textbf{Exp. Output:} The state is saved in a file and when the file is loaded, the saved state is restored.
	
	\item[\textlabel{/T140/}{T:140}] \textbf{\textlabel{Place edges and vertices}{T:140T}} \\
	\textbf{Input:} The tester places vertices and connects them with edges for both players without intersection. \\
	\textbf{Exp. Output:} The placed vertices and edges are displayed. Each turn the status displays which player's turn it is.
	
	\item[\textlabel{/T150/}{T:150}] \textbf{\textlabel{Place an edge across another one}{T:150T}} \\
	\textbf{Input:} Two vertices have been connected by an edge. The tester clicks on two other vertices on either side of the edge. \\
	\textbf{Exp. Output:} The vertices do not get connected. An error message is displayed.
	
	\item[\textlabel{/T160/}{T:160}] \textbf{\textlabel{Connect vertices of wrong distance}{T:160T}} \\
	\textbf{Input:} Two vertices have been placed that are not in a knight's move distance. The tester clicks them both. \\
	\textbf{Exp. Output:} The vertices do not get connected. An error message is displayed.
	
	\item[\textlabel{/T170/}{T:170}] \textbf{\textlabel{Place a vertex on an already occupied field}{T:170T}} \\
	\textbf{Input:} A vertex has been placed. The tester clicks the vertex again. \\
	\textbf{Exp. Output:} Nothing changes.
	
	\item[\textlabel{/T180/}{T:180}] \textbf{\textlabel{Win/Lose a game}{T:180T}} \\
	\textbf{Input:} The tester plays for both players until one wins the game. \\
	\textbf{Exp. Output:} A win message for the winning player is displayed.
\end{description}

\subsection{Manual test scenarios}

For details about the test scenarios please refer to section \emph{11.3 Test scenarios} of the functional specifications document.\par

\begin{tabular}{clll}

\hline
	\textbf{Scenario} & \textbf{Description} & \textbf{Result} & \textbf{Remark} \\
	\hline
	11.3.1 & Use the \gameexplorer & Passed & \\
	11.3.2 & Play \graphcoloring single-player & Failed & Implementation changed \\
	11.3.3 & Play \graphcoloring multiplayer & Passed & \\
	11.3.4 & Play \twixt & Not verified & \\
	\hline
\end{tabular}

\subsection{Code coverage}

\todo{Statistics about code coverage here.}