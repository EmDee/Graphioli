\usepackage[english]{babel}
\usepackage[utf8]{inputenc}
\usepackage[T1]{fontenc}
\usepackage{ae}
\usepackage[pdftex,bookmarks=true,bookmarksnumbered,colorlinks=true]{hyperref} % http://ctan.org/pkg/hyperref
\usepackage{nameref}
\usepackage{xparse}
\usepackage{parskip}
\usepackage{comment} % for block commenting
\usepackage{longtable} % longtable breaks longer tables to different pages
\usepackage[table,usenames,dvipsnames]{xcolor} % has to be loaded before tikZ
\usepackage{array}
\usepackage{etoolbox} % to set table rows to 0
%%%%%%% package for sequence diagrams
\usepackage{tikz}
\usetikzlibrary{arrows,shadows}
\usepackage{../materials/resources/pgf-umlsd}
%%%%%%%
\usepackage{float}
\graphicspath{{../materials/images/design/}}
\usepackage{todonotes} % use temporary for todo notes
% \usepackage{lineno} % temporary use for correcting
\usepackage{xspace}
% \linenumbers
\usepackage{lipsum} % just to generate some text
\usepackage{../materials/resources/dashrule} % for dashed rules
\usepackage[nonumberlist,acronym,toc,section]{glossaries} % must be the last package to include

\setlength{\parindent}{0cm}
\setlength{\parskip}{8pt}
\setlength{\LTpost}{0pt} % space after `longtable`

\widowpenalty=10000 % Prevent single line after pagebreak
\clubpenalty=10000  % Prevent single line before pagebreak

% rewrite long style for acronyms, thus it doesn't get in the way with the table coloring
% see original code in glossaries.dtx
\newglossarystyle{altlong}{%
\rowcolors{1}{}{}
  \renewenvironment{theglossary}%
     {\begin{longtable}{lp{\glsdescwidth}}}%
     {\end{longtable}}%
  \renewcommand*{\glossaryheader}{}%
  \renewcommand*{\glsgroupheading}[1]{}%
  \renewcommand*{\glossaryentryfield}[5]{%
    \glsentryitem{##1}\glstarget{##1}{##2} & ##3\glspostdescription\space ##5\\}%
  \renewcommand*{\glossarysubentryfield}[6]{%
     & 
     \glssubentryitem{##2}%
     \glstarget{##2}{\strut}##4\glspostdescription\space ##6\\}%
  \renewcommand*{\glsgroupskip}{ & \\}%
}

% glossary
\makeglossaries
% Remove trailing dot
\renewcommand*{\glspostdescription}{}

%
% Glossary entries
%

\newglossaryentry{graphioli}{name=Graphioli,
description={A fancy name that combines \glspl{graph} with filled pasta.},
plural={Graphioli}}

\newglossaryentry{graph}{name=graph,
description={A (abstract) set of objects provided with binary relations.\\The objects of a graph are called \glspl{vertex} and are represented by nodes in the visualization. The relations are called \glspl{edge} and are represented by (in our case straight) lines between the respective vertices in a visualized graph.\todo{Passt so?}}}

\newglossaryentry{edge}{name=edge,
description={see \emph{\gls{graph}}}}

\newglossaryentry{vertex}{name=vertex,
description={see \emph{\gls{graph}}},
plural={vertices}}

\newglossaryentry{planar}{name=planar,
description={A \emph{planar} \gls{graph} is a graph that can be structure-preservingly redrawn in a way that no \glspl{edge} intersect.},
plural={planar}}

\newglossaryentry{planarity}{name=planarity,
description={see \emph{\gls{planar}}},
plural={planarities}}

\newglossaryentry{api}{name=application programming interface (API),
description={Set of routines and tools providing \gls{program} blocks, which are put together by a third-party programmer to ensure consistency with an existing program or \gls{framework}.}}

\newglossaryentry{gui}{name=graphical user interface (GUI),
description={\Gls{program} interface that uses a computer's graphical capabilities to provide easy access and handling for the program. It replaces the command-line based access.},
plural={graphical user interfaces (GUI)}}

\newglossaryentry{framework}{name=framework,
description={A layered structure indicating what kind of applications can or should be built and how they would interrelate. It specifies interfaces and reusable \glspl{program} that form the basis for such an application.}}

\newglossaryentry{java}{name=Java,
description={Java is a widely spread programming language originally developed by Sun Microsystems and released in 1995. Applications developed with Java are compiled into bytecode that runs on basically every system that has a \Gls{jvm} installed.},
plural={Java}}

\newglossaryentry{jvm}{name=Java Virtual Machine (JVM),
description={A virtual machine that can execute \Gls{java} byte code. Thus, the execution component of the Java platform.},
plural={Java Virtual Machines (JVM)}}

\newglossaryentry{jdk}{name=Java Development Kit (JDK),
description={A collection of programming tools for \Gls{java} developers.},
plural={Java Development Kit (JDK)}}

\newglossaryentry{jre}{name=Java Runtime Environment,
description={see \emph{\gls{jvm}}}}

\newglossaryentry{javadoc}{name=Javadoc,
description={Javadoc is a tool for generating API documentation in HTML format from doc comments in source code.\footnote{From Oracle, self-definition}},
plural={Javadoc}}

\newglossaryentry{customer}{name=customer,
description={A party that receives a copy of the created \gls{software} in order to use it for own implementations. The customer of this \gls{framework} will be the \gls{developer}.}}

\newglossaryentry{computer-game}{name=computer game,
description={An electronic game that requires the \gls{player}['s] interaction to influence its course of events. Such a game usually bases on graphical feedback.}}

\newglossaryentry{game}{name=game,
description={The implementation of a \gls{computer-game} that requires the usage of this \gls{framework}.}}

\newglossaryentry{network}{name=network,
description={A group of connected computers that are able to communicate and share data, e.g. for synchronizing \gls{game} statuses.}}

\newglossaryentry{tutorial}{name=tutorial,
description={Instructional document that provides step by step information about a specific topic or \gls{workflow}, e.g. how to start implementing your own \gls{game} using this \gls{framework}.}}

\newglossaryentry{editor}{name=editor,
description={see \emph{\gls{level-editor}}}}

\newglossaryentry{level-editor}{name=level editor,
description={An interface that lightens the \gls{developer}['s] workload at creating new levels.}}

\newglossaryentry{level}{name=level,
description={A stage of the current \gls{game}.}}

\newglossaryentry{player}{name=player,
description={One who plays one of the \glspl{game} developed by a \gls{developer} using this \gls{framework}.}}

\newglossaryentry{developer}{name=developer,
description={Someone who designs and implements \glspl{game} using the {\graphioli} framework.}}

\newglossaryentry{undo}{name=undo,
description={To restore a previous condition, e.g. a \gls{savegame} or the previous step in the \gls{level-editor}.}}

\newglossaryentry{redo}{name=redo,
description={Opposite of \emph{\gls{undo}}.}}

\newglossaryentry{four-color-theorem}{name=four color theorem,
description={In mathematics, the \emph{four color theorem} states that, given any separation of a plane into contiguous areas, no more than four colors are required to color these areas in a way that no two \gls{adjacent} areas have the same color.},
plural={four color theorem}}

\newglossaryentry{adjacent}{name=adjacent,
description={Two \glspl{vertex} connected by an \gls{edge} are called \emph{adjacent}.},
plural={adjacent}}

\newglossaryentry{program}{name=program,
description={see \emph{\gls{computer-program}}}}

\newglossaryentry{savegame}{name=savegame,
description={A data file that contains information about a current \gls{game} state and is created when saving a game. It allows the \gls{player} to jump to a previously reached level.}}

\newglossaryentry{computer-program}{name=computer program,
description={A sequence of instructions that a computer can interpret and execute.}}

\newglossaryentry{software}{name=software,
description={An organized collection of data and instructions for computers, that is used to accomplish specific tasks.\\A piece of software can consists of a single \gls{program} or a package of programs closely working together. Usually software is bundled with associated documentation.}}

\newglossaryentry{workflow}{name=workflow,
description={The sequence of administrative, technical, or other processes through which a \gls{software} project passes from initiation to completion.}}

\newglossaryentry{user}{name=user,
description={The \emph{user} of a \gls{software} is everybody how starts the \gls{program} in order to fulfill specified tasks.\\The user of this framework is the \gls{developer}, whereas the user of the final \gls{game} is the \gls{player} (see \ref{REF:HOW-TO-READ}).}}

\newglossaryentry{git}{name=Git,
description={A distributed revision control and source code management system developed by \emph{Linus Torvalds}. Every Git working directory is a full-fledged repository with complete history and full revision tracking capabilities, not dependent on network access or a central server.\footnote{From Wikipedia: Git (software), 08/05/2012}},
plural={Git}}

\newglossaryentry{eclipse}{name=Eclipse,
description={An open development platform comprised of extensible frameworks, tools and runtimes for building, deploying and managing software across the lifecycle.\footnote{From Eclipse project, self-definition}},
plural={Eclipse}}

\newglossaryentry{artificial-intelligence}{name=artificial intelligence (AI),
description={An intelligent system that perceives its environment and takes actions that maximize its chances of success.\footnote{Russell, Stuart J.; Norvig, Peter: Artificial Intelligence: A Modern Approach (2003, 2nd ed.)}\\An algorithm that tries to defeat its human counter-\glspl{player} in a \glspl{computer-game} could be referred to as \emph{artificial intelligence}.}}

\newglossaryentry{depth-first-search}{name=depth-first search (DFS),
description={An \gls{algorithm} for traversing or searching a \gls{graph}, starting at a selected root \gls{vertex} and exploring as far as possible along each branch before backtracking.\footnote{From Wikipedia: Breadth-first search, 08/05/2012}},
plural={depth-first searches (DFS)}}

\newglossaryentry{breadth-first-search}{name=breadth-first search (BFS),
description={An \gls{algorithm} for traversing or searching a \gls{graph}, starting at a selected root \gls{vertex} and inspecting all its neighboring vertices.},
plural={breadth-first searches (BFS)}}

\newglossaryentry{path}{name=path,
description={A sequence of \glspl{vertex} of a \gls{graph}.}}

\newglossaryentry{algorithm}{name=algorithm,
description={Procedure or set of (mathematical) rules for solving a problem in a finite number of steps, especially by a computer.}}

\newglossaryentry{library}{name=library,
description={A collection of resources used to develop \gls{software}. These may include pre-written code and subroutines, classes, values or type specifications.},
plural={libraries}}

\newglossaryentry{opensuse}{name=openSUSE,
description={A free and \Gls{linux}-based operating system.\footnote{From openSUSE Project, self-definition}},
plural={openSUSE}}

\newglossaryentry{linux}{name=Linux,
description={An alternative and open-source operating system for personal computers and servers.},
plural={Linuces}}

\newglossaryentry{metalanguage}{name=metalanguage,
description={Any (technical) language used to describe and define an implementation of a \gls{game} in the \gls{framework}. It substitutes the \Gls{java}-based programming of that implementation and can be parsed by the framework.}}

\newglossaryentry{mvc}{name=Model-View-Controller (MVC),
description={A design pattern for computer user interfaces describing the structural subdivision into the components `Model', `View' and `Controller'.},
plural={Model-View-Controllers (MVC)}}

\newglossaryentry{chat}{name=online chat,
description={A communication system over the internet.}}

%
% Acronyms
%

\newacronym{GUI}{GUI}{graphical user interface\protect\glsadd{gui}}

\newacronym{API}{API}{application programming interface\protect\glsadd{api}}

\newacronym{JVM}{JVM}{Java Virtual Machine\protect\glsadd{jvm}}

\newacronym{JDK}{JDK}{Java Development Kit\protect\glsadd{jdk}}

\newacronym{AI}{AI}{artificial intelligence\protect\glsadd{artificial-intelligence}}

\newacronym{DFS}{DFS}{depth-first search\protect\glsadd{depth-first-search}}

\newacronym{BFS}{BFS}{breadth-first search\protect\glsadd{breadth-first-search}}

\newacronym{MVC}{MVC}{Model-View-Controller\protect\glsadd{mvc}}

\newacronym{IDE}{IDE}{integrated development environment}

\newacronym{ID}{ID}{identification}

% make use of row counter to be able to color "row mod #". In this case two at a time.
\rowcolors{1}{white}{gray!20}

\newcounter{row}
\newcolumntype{L}{%
  >{\stepcounter{row}%
  \pgfmathtruncatemacro{\j}{int(Mod(\therow,4))}
  \ifnum\j<2 
    \global\rownum=1
  \else
    \global\rownum=0
  \fi}%
l}

\BeforeBeginEnvironment{longtable}{\setcounter{row}{-1}} % set first row in table to -1 for title

% generates auto indented lists like in JavaDoc for inheritance hierarchy
\ExplSyntaxOn
\NewDocumentCommand{\createindentedlist}{m}
 {
  \parbox{\textwidth}{
    \skip_zero:N \parskip
    \skip_set:Nn \parindent { 1.5em }
    \cil_iterate:n { #1 }
    \bigskip
  }
 }
\cs_new_protected:Npn \cil_iterate:n #1
 {
  \int_set:Nn \cil_iteration_int { -1 }
  \clist_map_inline:nn { #1 }
   {
    \hspace*{\cil_iteration_int \parindent} \textcolor{NavyBlue}{\textsf{##1}}
    \int_incr:N \cil_iteration_int
    \par\nobreak
   }
 }
\int_new:N \cil_iteration_int
\ExplSyntaxOff