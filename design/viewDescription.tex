\subsection{View}

% Interface View
\interface{View}{view}

View is an interface for implementation of the (graphical) user interface. \\ 
\centerdash

\paragraph*{Method Summary}
\paragraph*{}
\begin{longtable}{Lp{10cm}}
	\startmethodtable
	\method{public boolean}{registerController(ViewManager controller)}{view:registercontroller} \\
	& Registers the controller for the user interface. \\
	\method{public boolean}{displayPopUp(String message)}{view:displaypopup} \\
	& Displays a PopUp message. \\
	\method{public boolean}{addCustomMenuItem(MenuItem item)}{view:addcustommenuitem} \\
	& Adds a custom menu item to the menu (here: \ref{cls:menubar}). \\ 
	\method{public boolean}{updatePlayerStatus(Player player)}{view:updateplayerstatus}\\
	& Updates the current player shown in the status bar (\ref{cls:statusbar}). \\
	\method{public boolean}{displayErrorMessage(String message)}{view:displayerrormessage} \\
	& Displays a error message in the status bar (\ref{cls:statusbar}). \\
	\method{public boolean}{redrawGraph()}{view:redrawgraph} \\
	& Redraws the graph displayed in the canvas (\ref{cls:graphcanvas}). \\ \hline
\end{longtable}

% GameWindow
\class{GameWindow}{gamewindow}
\createindentedlist{java.lang.Object, java.awt.Component, java.awt.Container, java.awt.Window, java.awt.Frame, javax.swing.JFrame, de.graphioli.view.GameWindow}

This class is the actual game window containing \ref{cls:graphcanvas}, \ref{cls:statusbar} and \ref{cls:menubar}. It handles keyboard inputs with \ref{cls:customkeydispatcher}. \\ 

\interfaces{cls:view}
\centerdash

\paragraph*{Method Summary}
\paragraph*{}
\begin{longtable}{Lp{10cm}}
	\startmethodtable
	\method{public ViewManager}{getViewManager()}{gw:getviewmanager} \\
	& Returns the  associated \ref{cls:viewmanager}. \\
	\method{public boolean}{onKeyRelease(int keyCode)}{gw:onkeyrelease} \\
	& Forwards the key input to the \ref{cls:viewmanager}. \\ \hline
\end{longtable}

% CustomKeyDispatcher
\class{CustomKeyDispatcher}{customkeydispatcher}

This class handles the keyboard input. \\ 

\begin{description}
	\item[All Implemented Interfaces] \hfill \\
	java.awt.KeyEventDispatcher
\end{description}
\centerdash

\paragraph*{Method Summary}
\paragraph*{}
\begin{longtable}{Lp{10cm}}
	\startmethodtable
	\method{public}{CustomKeyDispatcher(GameWindow gamewindow)}{ckd:customkeydispatcher} \\
	& Creates a CustomKeyDispatcher and registers its parent \ref{cls:gamewindow}. \\\hline
\end{longtable}

% GraphCanvas
\class{GraphCanvas}{graphcanvas}
\createindentedlist{java.lang.Object, java.awt.Component, java.awt.Container, javax.swing.JComponent, javax.swing.JPanel, de.graphioli.view.GraphCanvas}

This class is responsible for displaying the graph and handles mouse clicks with \ref{cls:visualgrid}. \\ 
\centerdash

\paragraph*{Method Summary}
\paragraph*{}
\begin{longtable}{Lp{10cm}}
	\startmethodtable
	\method{public}{GraphCanvas(GameWindow parent)}{gc:graphcanvas} \\
	& Creates a GraphCanvas and registers its parent \ref{cls:gamewindow}. \\
	\method{public boolean}{updateCanvas()}{gc:updatecanvas} \\
	& Updates and redraws the canvas. \\ \hline
\end{longtable}

% VisualGrid
\class{VisualGrid}{visualgrid}

This class handles mouse input and serves as an connector between \ref{cls:grid} and display. \\ 
\begin{description}
	\item[All Implemented Interfaces] \hfill \\
	java.awt.event.MouseListener
\end{description}
\centerdash

\paragraph*{Method Summary}
\paragraph*{}
\begin{longtable}{Lp{10cm}}
	\startmethodtable
	\method{public}{VisualGrid(GameWindow gamewindow)}{vg:visualgrid} \\
	& Creates a VisualGrid and registers its parent \ref{cls:gamewindow}. \\
	\method{private GridPoint}{parseCoordinates(int xCoord, int yCoord)}{vs:parsecoordinates} \\
	& Parses the coordinates of the mouse click to the specific \ref{cls:gridpoint}. \\
	\method{public boolean}{setVisualVertexSize(int size)}{vs:setvisualvertexsize} \\
	& Sets the size of the displayed vertices up to a grid specific maximum value. \\
	\method{public int}{getVisualVertexSize()}{vs:getvisualvertexsize} \\
	& Returns the size of the displayed vertices. \\ \hline
\end{longtable}

% MenuBar
\class{MenuBar}{menubar}
\createindentedlist{java.lang.Object, java.awt.Component, java.awt.Container, javax.swing.JComponent, javax.swing.JMenuBar, de.graphioli.view.MenuBar}

The MenuBar contains the game menu, options menu and help menu, is displayed at the top of the \ref{cls:gamewindow} and starts the standard menu item specific methods. \\ 
\centerdash

\paragraph*{Method Summary}
\paragraph*{}
\begin{longtable}{Lp{10cm}}
	\startmethodtable
	\method{public}{MenuBar(GameWindow parent)}{mb:menubar} \\
	& Creates the MenuBar and registers its parent \ref{cls:gamewindow}. \\
	\method{public boolean}{addCustomMenuItem(MenuItem item)}{mb:addcustommenuitem} \\
	& Adds a \ref{cls:menuitem} to the options menu. \\ \hline
\end{longtable}

%CustomMenuItemListener
\class{CustomMenuItemListener}{custommenuitemlistener}

Listens to one specific \ref{cls:menuitem} in the \ref{cls:menubar} and forwards it to the \ref{cls:viewmanager} if it is selected. \\
\begin{description}
	\item[All Implemented Interfaces] \hfill \\
		java.awt.event.ActionsListener
\end{description}
\centerdash

\paragraph*{Method Summary}
\paragraph*{}
\begin{longtable}{Lp{10cm}}
	\startmethodtable
	\method{public}{CustomMenuItemListener(GameWindow window, MenuItem item)}{cmil:custommenuitemlistener} \\
	& Creates a CustomMenuItemListener and registers its specific \ref{cls:menuitem} and the parent \ref{cls:gamewindow}. \\ \hline
\end{longtable}

% StatusBar
\class{StatusBar}{statusbar}
\createindentedlist{java.lang.Object, java.awt.Component, java.awt.Container, javax.swing.JComponent, javax.swing.JPanel, de.graphioli.view.StatusBar}

This class represents a status bar at the bottom of the \ref{cls:gamewindow} and displays the current player and short error messages. \\
\centerdash

\paragraph*{Method Summary}
\paragraph*{}
\begin{longtable}{Lp{10cm}}
	\startmethodtable
	\method{public String}{getStatus()}{sb:getstatus} \\
	& Returns the currently shown status message. \\
	\method{public boolean}{updatePlayerStatus(Player player)}{sb:updateplayerstatus} \\
	& Updates and displays the current player. \\
	\method{public boolean}{displayErrorMessage(String message)}{sb:displayerrormessage} \\
	& Updates and displays the current player. \\ \hline
\end{longtable}
