\subsection{Model}

% GameBoard
\class{GameBoard}{gameboard}
This class represents the game board of a game. The \texttt{GameBoard} combines the logical \ref{cls:graph} with the \ref{cls:grid} and has access to both of them. \\

\centerdash

\paragraph*{Method Summary}
\paragraph*{}
\begin{longtable}{Lp{10cm}}
	\startmethodtable
	\method{public}{GameBoard(boolean isDirected)}{gb:gameboard} \\
	& Creates a new \texttt{GameBoard} with the specified option - if the GameBoard is directed or not. \\
	\method{public boolean}{addVisualVertex(VisualVertex visualVertex)}{gb:addvisualvertex} \\
	& Adds a \ref{cls:visualvertex} to this \texttt{GameBoard} and returns \texttt{true} if adding was successful. \\
	\method{public boolean}{addVisualVertices(VisualVertex[ ] visualVertices)}{gb:addvisualvertices} \\
	& Adds a list of VisualVertices to this \texttt{GameBoard} and returns \texttt{true} if adding was successful. \\
	\method{public VisualEdge}{addVisualEdge(VisualVertex vertexA, VisualVertex vertexB)}{gb:addvisualedge} \\
	& Adds a \ref{cls:visualedge} between two specified \texttt{VisualVertices} \texttt{vertexA} and \texttt{vertexB}, and returns the added \ref{cls:visualvertex}. \\
	\method{public boolean}{removeVisualVertex(VisualVertex visualVertex)}{gb:removevisualvertex} \\
	& Returns \texttt{true} if a specified \ref{cls:visualvertex} was removed successfully from this \texttt{GameBoard}. \\
	\method{public boolean}{removeVisualEdge(VisualEdge visualEdge)}{gb:removevisualedge} \\
	& Returns \texttt{true} if a specified \ref{cls:visualedge} was removed successfully from this \texttt{GameBoard}. \\
	\method{public VisualEdge}{getVisualEdge(VisualVertex vexA, VisualVertex vexB)}{gb:getVisualEdge} \\
	& Returns a \ref{cls:visualedge} between two specified \texttt{VisualVertices} \texttt{vertexA} and \texttt{vertexB}. \\
	\method{public boolean}{isGraphDirected()}{gb:isgraphdirected} \\
	& Returns true if the \ref{cls:graph} in this \texttt{GameBoard} is directed. \\
	\method{public Graph}{getGraph()}{gb:getgraph} \\
	& Returns the \ref{cls:graph} in this \texttt{GameBoard}. \\
	\method{public Grid}{getGrid()}{gb:getgrid} \\
	& Returns the \ref{cls:grid} in this \texttt{GameBoard}. \\
	\hline
\end{longtable}

% Grid
\class{Grid}{grid}
This class represents the \texttt{Grid}, which the \ref{cls:graph} will be located on. \\

\centerdash

\paragraph*{Method Summary}
\paragraph*{}
\begin{longtable}{Lp{10cm}}
	\startmethodtable
	\method{public}{Grid(int horizontalGridPoints, int verticalGridPoints)}{grid:grid} \\
	& Creates a new \texttt{Grid} with the specified parameters. \\
	\method{public boolean}{addVisualVertexToGrid(VisualVertex visualVertex)}{grid:addvisualvertextogridpoint} \\
	& Returns \texttt{true} if a specified \ref{cls:visualvertex} was added successfully to this \texttt{Grid}. \\
	\method{public boolean}{removeVisualVertexAtGridPoint(GridPoint gridPoint)}{grid:removevisualvertexatgridpoint} \\
	& Returns \texttt{true} if a \ref{cls:visualvertex} was removed successfully from the specified \ref{cls:gridpoint} on this \texttt{Grid}. \\
	\method{public VisualVertex}{getVertexAtGridPoint(GridPoint gridPoint)}{grid:getvertexatgridpoint} \\
	& Returns a \ref{cls:visualvertex} from a specified \ref{cls:gridpoint} on this \texttt{Grid}. \\
	\hline
\end{longtable}

% GridPoint
\class{GridPoint}{gridpoint}
This class represents a \texttt{GridPoint} on a \texttt{Grid}. \\

\centerdash

\paragraph*{Method Summary}
\paragraph*{}
\begin{longtable}{Lp{10cm}}
	\startmethodtable
	\method{public}{GridPoint(int positionX, int positionY)}{gp:gridpoint} \\
	& Creates a new \texttt{GridPoint} with the specified positions \texttt{positionX} and \texttt{positionY}. \\
	\method{public int}{getPositionX()}{gp:getpositionx} \\
	& Returns the x coordinates of the \texttt{GridPoint}. \\
	\method{public int}{getPositionY()}{gp:getpositiony} \\
	& Returns the y coordinates of the \texttt{GridPoint}. \\
	\hline
\end{longtable}

% Graph
\class{Graph}{graph}
This class represents the logical graph. \\

\centerdash

\paragraph*{Method Summary}
\paragraph*{}
\begin{longtable}{Lp{10cm}}
	\startmethodtable
	\method{public boolean}{addEdge(Vertex vertexA, Vertex vertexB)}{graph:addEdge} \\
	& Adds an \ref{cls:edge} to this graph. \\
	\method{public boolean}{addVertex(vertex : Vertex)}{graph:addVertex} \\
	& Adds a \ref{cls:vertex} to this graph. \\
	\method{public boolean}{removeVertex(Vertex vertex)}{graph:removevertex} \\
	& Removes a specified \ref{cls:vertex} from this graph. \\
	\method{public boolean}{removeEdge(Edge edge)}{graph:removeedge} \\
	& Removes a specified \ref{cls:edge} from this graph. \\
	\method{public Iterable<Vertex>}{getVertices()}{graph:iterablevertex} \\
	& Returns an iterable list of vertices.  \\
	\method{public Iterable<Edge>}{getEdge()}{graph:iterableedge} \\
	& Returns an iterable list of vertices. \\
	\hline
\end{longtable}

% Vertex
\class{Vertex}{vertex}
This class represents the logical vertex. \\

\subclasses{cls:visualvertex}

\centerdash

\paragraph*{Method Summary}
\paragraph*{}
\begin{longtable}{Lp{10cm}}
	\startmethodtable
	\method{public Iterable<Vertex>}{getAdjacentVertices()}{vertex:getadjacentvertices} \\
	& Returns all vertices that are connected to this vertex. \\
	\method{public UUID}{getUID()}{vertex:getuid} \\
	& Returns the unique identifier for this vertex. \\
	\method{public Iterable<Edge>}{getIncomingEdges()}{vertex:getincomingedges} \\
	& Returns an iterable list of incoming edges. \\
	\method{public boolean}{addIncomingEdge(Edge edge)}{vertex:addincomingedge} \\
	& Adds an incoming \ref{cls:edge} to the list of incoming edges. \\
	\method{public Iterable<Edge>}{getOutgoingEdges()}{vertex:getoutgoingedges} \\
	& Returns an iterable list of outgoing edges. \\
	\method{public boolean}{addOutgoingEdge(Edge edge)}{vertex:addoutgoingedge} \\
	& Adds an outgoing \ref{cls:edge} to the list of outgoing edges. \\
	\method{public boolean}{isAdjacentTo(Vertex vertex)}{vertex:isadjacentto} \\
	\hline
\end{longtable}

% VisualVertex
\abstractclass{VisualVertex}{visualvertex}
\createindentedlist{de.graphioli.model.Vertex, de.graphioli.model.VisualVertex}
This class represents a VisualVertex. \\

\subclasses{cls:graphicvisualvertex, cls:simplevisualvertex}

\centerdash

\paragraph*{Method Summary}
\paragraph*{}
\begin{longtable}{Lp{10cm}}
	\startmethodtable
	\method{public}{VisualVertex(gridPoint : GridPoint)}{vv:visualvertex} \\
	& Creates a new VisualVertex and places it on the specified \ref{cls:gridpoint}. \\
	\method{protected abstract void}{draw(Graphics2D)}{vv:draw} \\
	& Draws the VisualVertex. \\
	\method{final void}{update()}{vv:update} \\
	& Updated the VisualVertex. \\ 
	\method{public \ref{cls:gridpoint}}{getGridPoint()}{vv:getgridpoint} \\
	& Returns the GridPoint of the VisualVertex. \\ 
	\method{public boolean}{setGridPoint(gridPoint : GridPoint)}{vv:setGridPoint(gridPoint : GridPoint)} \\
	& Sets the GridPoint of the VisualVertex. \\
	\method{public BufferedImage}{getBufferedImage()}{vv:getbufferedimage} \\
	& Returns the BufferedImage on the VisualVertex. \\ 
	\hline
\end{longtable}

% SimpleVisualVertex
\class{SimpleVisualVertex}{simplevisualvertex}
\createindentedlist{de.graphioli.model.Vertex, de.graphioli.model.VisualVertex, de.graphioli.model.SimpleVisualVertex}
This class represents a simple \ref{cls:visualvertex} with predefined attributes.\\

\subclasses{cls:graphcoloringvertex}

\centerdash

\paragraph*{Method Summary}
\paragraph*{}
\begin{longtable}{Lp{10cm}}
	\startmethodtable
	\method{public Color}{getColor()}{svv:getcolor} \\
	& Returns the color of this \ref{cls:simplevisualvertex}. \\
	\method{public int}{getStrokeWeight()}{svv:getStrokeWeight} \\
	& Returns the stroke weight of this \ref{cls:simplevisualvertex}. \\
	\method{public Color}{getStrokeColor()}{svv:getstrokecolor} \\
	& Returns the stroke color of this \ref{cls:simplevisualvertex}. \\
	\method{public boolean}{setColor(Color color)}{svv:setcolor} \\
	& Sets the color of this \ref{cls:simplevisualvertex}. \\
	\method{public boolean}{setStrokeWeight(int strokeWeight)}{svv:setstrokeweight} \\
	& Sets the stroke weight of this \ref{cls:simplevisualvertex}. \\
	\method{public boolean}{setStrokeColor(Color strokeColor)}{svv:setstrokecolor} \\
	& Sets the stroke color of this \ref{cls:simplevisualvertex}. \\
	\hline
\end{longtable}

% GraphColoringVertex
\class{GraphColoringVertex}{graphcoloringvertex}
\createindentedlist{de.graphioli.model.Vertex, de.graphioli.model.VisualVertex, de.graphioli.model.SimpleVisualVertex, de.graphioli.model.GraphColoringVertex}
This class represents a \texttt{GraphColoringVertex}. \\

\centerdash

\paragraph*{Method Summary}
\paragraph*{}
\begin{longtable}{Lp{10cm}}
	\startmethodtable
	\method{public int}{getColorID}{gcv:getcolorid} \\
	& Returns the \texttt{color identification} of this \texttt{GraphColoringVertex}. \\
	\method{public boolean}{isButton()}{gcv:isbutton} \\
	& Returns \texttt{true} if this \texttt{GraphColoringVertex} is a button. \\
	\method{public boolean}{isHighlighted()}{gcv:ishighlighted} \\
	& Returns \texttt{true} if this \texttt{GraphColoringVertex} is selected. \\
	\method{public boolean}{setColorID(int id)}{gcv:setcolorid} \\
	& Returns \texttt{true} if the placement of the specified \texttt{color identification} for this \texttt{GraphColoringVertex} was successful. \\
	\method{public boolean}{setButton(boolean isButton)}{gcv:setbutton} \\
	& Sets this \texttt{GraphColoringVertex} as \texttt{button}. \\
	\method{public boolean}{setHighlighted(boolean isHighlighted)}{gcv:sethighlighted} \\
	& Sets this \texttt{GraphColoringVertex} as selected. \\
	\hline
\end{longtable}

% GraphicVisualVertex
\class{GraphicVisualVertex}{graphicvisualvertex}
\createindentedlist{de.graphioli.model.Vertex, de.graphioli.model.VisualVertex, de.graphioli.model.GraphicVisualVertex}
This class represents a \ref{cls:visualvertex} with a custom icon. \\

\subclasses{cls:twixtvertex}

\centerdash

\paragraph*{Method Summary}
\paragraph*{}
\begin{longtable}{Lp{10cm}}
	\startmethodtable
	\method{public}{GraphicVisualVertex(String fileName)}{gvv:graphicvisualvertexfile} \\
	& Creates a new \texttt{GraphicVisualVertex} with the specified url. \\
	\method{public}{GraphicVisualVertex(BufferedImage image)}{gvv:graphicvisualverteximage} \\
	& Creates a new \texttt{GraphicVisualVertex} with the specified BufferedImage. \\
	\method{public BufferedImage}{loadBufferedImage(String fileName)}{gvv:loadbufferedimage} \\
	& Loads the image file at specified \texttt{fileName} and returns a \texttt{BufferedImage}. \\
	\method{public BufferedImage}{getImage()}{gvv:getimage} \\
	& Returns the BufferedImage that represents this \ref{cls:graphicvisualvertex}. \\
	\method{public boolean}{setImage(BufferedImage image)}{gvv:setimage} \\
	& Sets the BufferedImage that represents this \ref{cls:graphicvisualvertex}. \\ 
	\hline
\end{longtable}

% TwixtVertex
\class{TwixtVertex}{twixtvertex}
\createindentedlist{de.graphioli.model.Vertex, de.graphioli.model.VisualVertex, de.graphioli.model.GraphicVisualVertex,  de.graphioli.model.TwixtVertex}
This class represents a \texttt{TwixtVertex}. \\

\centerdash

\paragraph*{Method Summary}
\paragraph*{}
\begin{longtable}{Lp{10cm}}
	\startmethodtable
	\method{public}{TwixtVertex(int playerID)}{tv:twixtvertex} \\
	& Creates a new \texttt{TwixtVertex} and associate it with a \texttt{player} with the specified \texttt{playerID}. \\
	\method{public int}{getPlayerID()}{tv:getplayerid} \\
	& Returns the \ref{Player}'s identification. \\
	\method{public boolean}{init()}{tv:init} \\
	& Initializes this \texttt{TwixtVertex}. \\
	\hline
\end{longtable}

% Edge
\class{Edge}{edge}
This class represents an \texttt{Edge}. \\

\subclasses{cls:visualedge}

\centerdash

\paragraph*{Method Summary}
\paragraph*{}
\begin{longtable}{Lp{10cm}}
	\startmethodtable
	\method{public}{Edge(Vertex vertexA, Vertex vertexB)}{edge:edge} \\
	& Creates a new \texttt{Edge} with the specified Vertices \texttt{vertexA} and \texttt{vertexB}. \\
	\method{public Vertex}{getOriginVertex()}{edge:getoriginvertex} \\
	& Returns the origin \ref{cls:vertex}. \\
	\method{public Vertex}{getTargetVertex()}{edge:gettargetvertex} \\
	& Returns the end \ref{cls:vertex}. \\
	\method{public int}{getWeight()}{edge:getweight} \\
	& Returns the weight of this \texttt{edge}. \\
	\method{public boolean}{setWeight(int weight)}{edge:setweight} \\
	& Sets the weight of this \texttt{edge}. \\
	\hline
\end{longtable}

% VisualEdge
\abstractclass{VisualEdge}{visualedge}
\createindentedlist{de.graphioli.model.Edge, de.graphioli.model.VisualEdge}
This class represents a \texttt{VisualEdge}.

\subclasses{cls:simplevisualedge}

% SimpleVisualEdge
\class{SimpleVisualEdge}{simplevisualedge}
\createindentedlist{de.graphioli.model.Edge, de.graphioli.model.VisualEdge, de.graphioli.model.SimpleVisualEdge}
This class represents an \texttt{Edge} with different attributes (color, stroke and stroke weight). \\

\centerdash

\paragraph*{Method Summary}
\paragraph*{}
\begin{longtable}{Lp{10cm}}
	\startmethodtable
	\method{public Color}{getColor()}{sve:getcolor} \\
	& Returns the \texttt{color} of this \texttt{SimpleVisualEdge}. \\
	\method{public int}{getStrokeWeight()}{sve:getstrokeweight} \\
	& Returns the \texttt{stroke weight} of this \texttt{SimpleVisualEdge}. \\
	\method{public boolean}{setColor(Color color)}{sve:setcolor} \\
	& Sets the \texttt{color} of this \texttt{SimpleVisualEdge}. \\
	\method{public int}{setStrokeWeight()}{sve:setstrokeweight} \\
	& Sets the \texttt{stroke weight} of this \texttt{SimpleVisualEdge}. \\
	\hline
\end{longtable}

% Player
\abstractclass{Player}{player}
This abtract class represents a \texttt{player}. \\

\subclasses{cls:localplayer}

\centerdash

\paragraph*{Method Summary}
\paragraph*{}
\begin{longtable}{Lp{10cm}}
	\startmethodtable
	\method{public}{Player(String name)}{player:player} \\
	& Creates a new \texttt{Player} with the specified \texttt{name}. \\
	\method{public UUID}{getUID()}{player:getuid} \\
	& Returns the \texttt{UUID} of this \texttt{Player}. \\
	\method{public String}{getName()}{player:getname} \\
	& Returns the \texttt{name} of this \texttt{Player}. \\
	\hline
\end{longtable}

% LocalPlayer
\class{LocalPlayer}{localplayer}
\createindentedlist{de.graphioli.model.Player, de.graphioli.model.LocalPlayer}
This class represents a \texttt{LocalPlayer}.

% GameCapsule
\class{GameCapsule}{gamecapsule}
This class represents a \texttt{GameCapsule} that holds all the important data, which are absolutely neccessary. A \gls{savegame} will be saved in such a \texttt{GameCapsule}, which later on can be loaded to resume a game.

\centerdash

\paragraph*{Method Summary}
\paragraph*{}
\begin{longtable}{Lp{10cm}}
	\startmethodtable
	\method{public}{GameCapsule(GameBoard board, Iterable<Player> player, Player activePlayer)}{gc:gamecapsule} \\
	& Creates a new \texttt{GameCapsule} with the specified parameters. \\
	\method{public GameBoard}{getBoard()}{gc:getboard} \\
	& Returns the \ref{cls:gameboard} of this \texttt{GameCapsule}. \\
	\method{public Iterable<Player>}{getPlayers()}{gc:getplayers} \\
	& Returns an iterable list of \ref{cls:player}s of this \texttt{GameCapsule}. \\
	\method{public Player}{getActivePlayer}{gc:getActivePlayer} \\
	& Returns the current active \ref{cls:player} of this \texttt{GameCapsule}. \\
	\hline
\end{longtable}

\pagebreak

% GameDefinition
\class{GameDefinition}{gamedefinition}
This class represents the game's definition. All crucial information that is needed before a game can be started will be saved in a new instance of this \texttt{GameDefinition}.

\centerdash

\paragraph*{Method Summary}
\paragraph*{}
\begin{longtable}{Lp{10cm}}
	\startmethodtable
	\method{public}{GameDefinition({params})}{gd:gamedefinition} \\
	& Creates a new \texttt{GameDefinition} with the specified parameters. \\
	\method{public String}{getName()}{gd:getname} \\
	& Returns the \texttt{name} of this \texttt{GameDefinition}, which is the name of the game. \\
	\method{public int}{getMaxPlayerCount()}{gd:getmaxplayercount} \\
	& Returns the maximum number of players for this specific game. \\
	\method{public int}{getMinPlayerCount()}{gd:getminplayercount} \\
	& Returns the minimum number of players for this specific game. \\
	\method{public String}{getGamePath()}{gd:getgamepath} \\
	& Returns the path of this specific game. \\ \todo{brauchen wir getgamepath, wenn wir schon den namen von dem Spiel haben?}
	\method{public String}{getDescription()}{gd:getdescription} \\
	& Returns the description of this specific game. \\
	\method{public String}{getFullyQualifiedClassName()}{gd:getfullyqualifiedclassname} \\
	& Returns the exact file name of this specific game within the \texttt{.jar} file. \\
	\method{public BufferedImage}{getScreenshot()}{gd:getscreenshot} \\
	& Returns a \texttt{BufferedImage} that is this game's screenshot. \\
	\method{public File}{getLocalizedString()}{gd:getlocalizedstring} \\
	& Returns a \texttt{String} telling which language file should be used. \\
	\method{public URI}{getHelpFile()}{gd:gethelpfile} \\
	& Returns the \texttt{URI} that links to the help file of this game. \\
	\method{public Iterable<Player>}{getPlayers()}{gd:getplayers} \\
	& Returns an iterable list of \ref{cls:player}s that will be used to initialize \ref{cls:player}s. \\
	\method{public Iterable<MenuItem>}{getMenu()}{gd:getmenu} \\
	& Returns an iterable list of \texttt{MenuItem}s that will be used to generate the \texttt{MenuBar}. \\
	\method{public int}{getHorizontalGridPointCount()}{gd:gethorizontalgridpointcount} \\
	& Returns the number of horizontal \ref{cls:gridpoint}s for this specific game. \\
	\method{public int}{getVerticalGridPointCount()}{gd:getverticalgridpointcount} \\
	& Returns the number of vertical \ref{cls:gridpoint}s for this specific game. \\
	\hline
\end{longtable}