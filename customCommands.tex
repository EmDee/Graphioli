% use this \textlabel command to have a reference within normal text; don't use any other referencing with textlabel enabled
\makeatletter
    \newcommand*{\textlabel}[2]{%
        \edef\@currentlabel{#1}% Set target label
        \leavevmode\phantomsection% Correct hyper reference link
        #1\label{#2}% Print and store label
    }
\makeatother

% proper nouns
\newcommand{\graphioli}{\textsc{Graphioli}\xspace}
\newcommand{\twixt}{\textsc {TwixT}\xspace}
\newcommand{\graphcoloring}{\textsc {Graph-Coloring}\xspace}
\newcommand{\gameexplorer}{\textsc {Game-Explorer}\xspace}

% creates the class name with its reference
\newcommand{\class}[2]{\subsubsection*{\textcolor{Blue}{Class \textlabel{#1}{cls:#2}\index{#1}}}}

% creates the interface name with its reference
\newcommand{\interface}[2]{\subsubsection*{\textcolor{Blue}{Interface \textlabel{#1}{cls:#2}\index{#1}}}}

% creates static with its reference
\newcommand{\static}[2]{\subsubsection*{\textcolor{Blue}{Static Class \textlabel{#1}{cls:#2}\index{#1}}}}

% creates the abstract class name with its reference
\newcommand{\abstractclass}[2]{\subsubsection*{\textcolor{Blue}{Abstract Class \textlabel{#1}{cls:#2}\index{#1}}}}

% creates a reference for all implemented interfaces
\newcounter{iref}
\newcommand{\refi}[1]{\def\do##1{\stepcounter{iref}\ifnumgreater{\value{iref}}{1}{, }{}\ref{##1}}\docsvlist{#1}}

\newcommand{\interfaces}[1]{
	\begin{description}
		\item[All Implemented Interfaces] \hfill \\
		\refi{#1}
	\end{description}
	\setcounter{iref}{0}
}

% creates a reference for all direct subclasses
\newcounter{sref}
\newcommand{\reff}[1]{\def\do##1{\stepcounter{sref}\ifnumgreater{\value{sref}}{1}{, }{}\ref{##1}}\docsvlist{#1}}

\newcommand{\subclasses}[1]{
	\begin{description}
		\item[Direct Known Subclasses] \hfill \\
		\reff{#1}
	\end{description}
	\setcounter{sref}{0}
}

% create centered dashed rule
\newcommand{\centerdash}{
	\begin{center}
		\hdashrule{14cm}{1pt}{1pt}
	\end{center}
}

% to easily implement methods
\newcommand{\method}[3]{#1 & \textcolor{NavyBlue}{\textlabel{#2}{#3}}} 

% creates the beginning of the method description table
\newcommand{\startmethodtable}{\hline\rowcolor{white}\textbf{Modifier and Type} & \textbf{Method and Description} \\ \hline}

% added, removed and minor changed methods
\newcommand{\newmethod}[3]{\rowcolor{green!10} + & #1 & #2 & #3}
\newcommand{\removedmethod}[3]{\rowcolor{red!10} - & #1 & #2 & #3}
\newcommand{\alteredmethod}[3]{\rowcolor{blue!10} = & #1 & #2 & #3}