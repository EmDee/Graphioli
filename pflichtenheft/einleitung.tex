\section{Introduction}
\Glspl{game} premised on \gls{graph} \glspl{algorithm} have always been played as a pastime. Mostly without the knowledge about the underlying mathematical theories, which are – of course – not always relevant for clearing \gls{level} after level. While two humans playing a graph game usually agree on the eventuation of one player winning the game, the same game on a computer must be computed to declare one \gls{player} the winner.\par

This is where {\graphioli} takes effect. The \Gls{java} \gls{framework} simplifies the process of developing graph-based computer games by providing a straightforward, intuitive \gls{library} that allows its user to focus the effort to the game's logical implementation. Recurring and redundant operations, algorithms and codings will be taken care of by {\graphioli}.\par

The framework still does not release the \gls{developer} from actual programming work. However, it reduces the amount of time formerly spent realizing graphical outputs, structures and user interactions, thus reinventing the wheel.\par

Products based on {\graphioli} will be well-structured, lightweight in terms of code size and easy to maintain.\par

\subsection{How to read this document}

In this \emph{Functional Specification Document} it is the talk of \glspl{customer}, \glspl{developer} and \glspl{player}. The customer of this product, the {\graphioli} framework, is the \emph{developer}. Someone who needs a library to implement his or her ideas of graph-based \glspl{game}.\par
The user who afterwards executes the program containing this implemented game is then referred to as the \emph{player}.