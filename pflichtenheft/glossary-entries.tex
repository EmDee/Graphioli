% Remove trailing dot
\renewcommand*{\glspostdescription}{}

%
% Glossary entries
%

\newglossaryentry{graphioli}{name=Graphioli,
description={A fancy name that combines \glspl{graph} with filled pasta.}}

\newglossaryentry{graph}{name=graph,
description={Abstract representation of a set of edges where some pairs of the \glspl{vertice} are connected by \glspl{edge}.}}

\newglossaryentry{edge}{name=edge,
description={Line that connects a pair of \glspl{vertice} in a \gls{graph}.}}

\newglossaryentry{vertice}{name=vertice,
description={Node in a \gls{graph}. Can but must not be connected with other \glspl{vertice} by an \gls{edge}.}}

\newglossaryentry{planar}{name=planar,
description={A \emph{planar} \gls{graph} is a graph that can be structure-preservingly redrawn in a way that no \glspl{edge} intersect.}}

\newglossaryentry{planarity}{name=planarity,
description={see{planar}}}

\newglossaryentry{api}{name=application programming interface (API),
description={Set of routines and tools providing \gls{program} blocks, which are put together by a third-party programmer to ensure consistency with an existing program or \gls{framework}.}}

\newglossaryentry{gui}{name=graphical user interface (GUI),
description={\Gls{program} interface that uses a computer's graphical capabilities to provide easy access and handling for the program. It replaces the command-line based access.}}

\newglossaryentry{framework}{name=framework,
description={DESCRIPTION}}

\newglossaryentry{java}{name=Java,
description={Java is a widely spread programming language originally developed by Sun Microsystems and released in 1995. Applications developed with Java are compiled into bytecode that runs on basically every system that has a Java Virtual Machine installed.}}

\newglossaryentry{jvm}{name=Java Virtual Machine (JVM),
description={DESCRIPTION}}

\newglossaryentry{javadoc}{name=JavaDoc,
description={DESCRIPTION}}

\newglossaryentry{customer}{name=customer,
description={A party that receives a copy of the created \gls{software} in order to use it for own implementations.}}

\newglossaryentry{computergame}{name=computer game,
description={DESCRIPTION}}

\newglossaryentry{game}{name=game,
description={DESCRIPTION}}

\newglossaryentry{network}{name=network,
description={A group of connected computers that are able to communicate and share data, e.g. for synchronizing \gls{game} statuses.}}

\newglossaryentry{tutorial}{name=tutorial,
description={Instructional document that provides step by step information about a specific topic or \gls{workflow}, e.g. the usage and initialization of a \gls{program}.}}

\newglossaryentry{editor}{name=editor,
description={DESCRIPTION}}

\newglossaryentry{leveleditor}{name=level editor,
description={DESCRIPTION}}

\newglossaryentry{level}{name=level,
description={DESCRIPTION}}

\newglossaryentry{player}{name=player,
description={DESCRIPTION}}

\newglossaryentry{developer}{name=developer,
description={DESCRIPTION}}

\newglossaryentry{undo}{name=undo,
description={DESCRIPTION}}

\newglossaryentry{program}{name=program,
description={see{computer-program}}}

\newglossaryentry{computer-program}{name=computer program,
description={A sequence of instructions that a computer can interpret and execute.}}

\newglossaryentry{software}{name=software,
description={An organized collection of data and instructions for computers, that is used to accomplish specific tasks.\\A piece of software can consists of a single \gls{program} or a package of programs closely working together. Usually software is bundled with associated documentation.}}

\newglossaryentry{workflow}{name=workflow,
description={DESCRIPTION}}

\newglossaryentry{consistency}{name=consistency,
description={Those attributes of the \gls{software} that provide uniform design and implementation techniques and notations.  [McCall, Richards and Walters, RADC, 1977]}}

\newglossaryentry{correctability}{name=correctability,
description={The degree of effort required to correct \gls{software} defects and to cope with \gls{user} complaints.  [Fenton \& Pfleeger, 1997]}}

\newglossaryentry{user}{name=user,
description={DESCRIPTION}}

\newglossaryentry{stability}{name=stability,
description={DESCRIPTION}}

\newglossaryentry{usability}{name=usability,
description={DESCRIPTION}}

\newglossaryentry{performance}{name=performance,
description={DESCRIPTION}}

\newglossaryentry{git}{name=Git,
description={A distributed revision control and source code management system developed by \emph{Linus Torvalds}. Every Git working directory is a full-fledged repository with complete history and full revision tracking capabilities, not dependent on network access or a central server.  [Wikipedia: Git (software), 08/05/2012]}}

\newglossaryentry{eclipse}{name=Eclipse,
description={DESCRIPTION}}

\newglossaryentry{artificial-intelligence}{name=artificial intelligence (AI),
description={DESCRIPTION}}

\newglossaryentry{depth-first-search}{name=depth-first search (DFS),
description={DESCRIPTION}}

\newglossaryentry{breadth-first-search}{name=breadth-first search (BFS),
description={DESCRIPTION}}

\newglossaryentry{path}{name=path,
description={DESCRIPTION}}

\newglossaryentry{algorithm}{name=algorithm,
description={Procedure or set of (mathematical) rules for solving a problem in a finite number of steps, especially by a computer.}}

%
% Acronyms
%

\newacronym{GUI}{GUI}{Graphical user interface\protect\glsadd{gui}}

\newacronym{API}{API}{Application programming interface\protect\glsadd{api}}

\newacronym{JVM}{JVM}{Java Virtual Machine\protect\glsadd{jvm}}

\newacronym{AI}{AI}{Artificial intelligence\protect\glsadd{artificial-intelligence}}

\newacronym{DFS}{DFS}{Depth-first search\protect\glsadd{depth-first-search}}

\newacronym{BFS}{BFS}{Breadth-first search\protect\glsadd{breadth-first-search}}