\section{Objectives}
\subsection{Mandatory criteria}
\subsubsection{Framework}
\begin{tabular}{{c}{l}}
    \hline
    {\bf Criteria} & {\bf Description} \\ \hline
	\ref{MC:FW010} & Creating graphs \\
	\ref{MC:FW020} & Displaying graphs \\
	\ref{MC:FW030} & Manipulating graphs \\
	\ref{MC:FW040} & Edges are equipped with parameters \\
	\ref{MC:FW050} & Vertices are equipped with parameters \\
	\ref{MC:FW060} & Basic graph algorithms \\
	\ref{MC:FW070} & Local multiplayer mode \\
	\ref{MC:FW080} & In-depth documentation \\
	\ref{MC:FW090} & Savegames \\ \hline
\end{tabular}

\vspace{1cm}

\begin{description}
  	\item[/FW010/\label{MC:FW010}] {\bf Creating graphs}  \hfill \\
  	Specific graphs can be created.
 	\item[/FW020/\label{MC:FW020}] {\bf Displaying graphs}  \\
 	Existing graphs can be visualized.
	\item[/FW030/\label{MC:FW030}] {\bf Manipulating graphs}  \\
  	Existing graphs can be modified (e.g. adding, removing or moving of vertices or edges)
	\item[/FW040/\label{MC:FW040}] {\bf Edges are equipped with parameters}  \\
	These parameters are at least style (background color, border color, background image, size), coordinates, weight, sorting, ID
	\item[/FW050/\label{MC:FW050}] {\bf Vertices are equipped with parameters}  \\
	These parameters are at least weight, length, direction
	\item[/FW060/\label{MC:FW060}] {\bf Basic graph algorithms}  \\
	These algorithms are at least path recognition (depth-first search, breadth-first search), Dijkstra.
	\item[/FW070/\label{MC:FW070}] {\bf Local multiplayer mode}  \\
	Two players are able to play against each other on the same computer sharing the same peripherals.
	\item[/FW080/\label{MC:FW080}] {\bf In-depth documentation}  \\
	Elaborate Javadoc documentation for the viewpoint of the developer using the framework.
	\item[/FW090/\label{MC:FW090}] {\bf Savegames}  \\
	Games can be saved and restored at a later time without losing level stats.
\end{description}

\subsubsection{GUI: Game-Explorer}
\begin{tabular}{{c}{l}}
    \hline
    {\bf Criteria} & {\bf Description} \\ \hline
	\ref{FR:GE010} & Load games \\
	\ref{FR:GE020} & List available games \\
	\ref{FR:GE030} & Displays description and screenshot to every game \\
	\ref{FR:GE040} & Start new or restart saved games \\
	\ref{FR:GE050} & Has a \'START\' button \\
\end{tabular}

\vspace{1cm}

\begin{description}
  	\item[/FR010/\label{FR:GE010}] {\bf Load games}  \hfill \\
  	Available games will be read dynamically from a subfolder.
 	\item[/FR020/\label{FR:GE020}] {\bf List available games}  \\
 	...
	\item[/FR030/\label{FR:GE030}] {\bf Displays description and screenshot to every game}  \\
  	...
	\item[/FR040/\label{FR:GE040}] {\bf Start new or restart saved games}  \\
	...
	\item[/FR050/\label{FR:GE050}] {\bf Has a \'START\' button}  \\
	...
\end{description}

\subsubsection{Games}
\begin{tabular}{{c}{l}}
    \hline
    {\bf Game} & {\bf Name} \\ \hline
	\ref{FR:GA010} & Graph coloring \\
	\ref{FR:GA020} & TwixT \\
\end{tabular}

\vspace{1cm}

\begin{description}
  	\item[/GA010/\label{FR:GA010}] {\bf Graph coloring}  \hfill \\
  	...
 	\item[/GA020/\label{FR:GA020}] {\bf TwixT}  \\
 	...
\end{description}


\subsection{Facultative criteria}
\subsubsection{Framework}
\subsubsection{GUI: Game-Explorer}
\subsubsection{Games}

\subsection{Excluded criteria}