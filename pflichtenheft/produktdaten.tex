\section{Product data}

\begin{tabular}{{c}{l}} \hline
    {\bf Data} & {\bf Description} \\ \hline
    \ref{PD:010} & \ref{PD:T010} \\
    \ref{PD:020} & \ref{PD:T020} \\
    \ref{PD:030} & \ref{PD:T030} \\
    \ref{PD:040} & \ref{PD:T040} \\ \hline
\end{tabular}

\begin{description}
	\item[\textlabel{/PD010/}{PD:010}] {\bf \textlabel{\explorer}{PD:T010}} \\
	In the \explorer there is a short description and screenshot of each game possible, given by the developer.
	\item[\textlabel{/PD020/}{PD:020}] {\bf \textlabel{Help}{PD:T020}} \\
	The developers provide an explanation and/or tutorial of their implemented graph games which can be accessed through the help function of the \explorer and game.
	\item[\textlabel{/PD030/}{PD:030}] {\bf \textlabel{Default designed levels (graphs)}{PD:T030}} \\
	Some games require different graphs each time they are played. The developer creates possible graphs to play his game which are then loaded as different levels.
	\item[\textlabel{/PD040/}{PD:040}] {\bf \textlabel{Savegames}{PD:T040}} \\
	Players are able to, if implemented by the developer, save their current state of the game. The data is saved in form of a text file in the folder of the respective game. \todo{Aufbau, wo ist die save file, screenshot, help-function-file, description???}
\end{description}