\section{Product data}

\begin{tabular}{{c}{l}} \hline
    {\bf Data} & {\bf Description} \\ \hline
    \ref{PD:010} & \ref{PD:T010} \\
    \ref{PD:020} & \ref{PD:T020} \\
    \ref{PD:030} & \ref{PD:T030} \\
    \ref{PD:040} & \ref{PD:T040} \\
    \ref{PD:050} & \ref{PD:T050} \\ \hline
\end{tabular}

\begin{description}
	\item[\textlabel{/PD010/}{PD:010}] {\bf \textlabel{Game-Explorer}{PD:T010}} \\
	The Game-Explorer displays a short description and screenshot of each \gls{game}, prepared by the \gls{developer}.
	\item[\textlabel{/PD020/}{PD:020}] {\bf \textlabel{Help}{PD:T020}} \\
	The developers provide an explanation or \gls{tutorial} of their implemented graph games which can be accessed through the help function of the Game-Explorer and within the game (\ref{FR:110}).
	\item[\textlabel{/PD030/}{PD:030}] {\bf \textlabel{Default designed levels (graphs)}{PD:T030}} \\
	Some games require different graphs each time they are played. The developer creates possible graphs to play his game which are then loaded as different \glspl{level}.
	\item[\textlabel{/PD040/}{PD:040}] {\bf \textlabel{Savegames}{PD:T040}} \\
	\Glspl{player} are able to, if implemented by the developer, save their current state of the game (\ref{FR:140}). The data is saved in form of a text file in the folder of the respective game. Those files can then be loaded by the player afterwards (\ref{FR:150}).
	\item[\textlabel{/PF050/}{PD:050}] {\bf \textlabel{Screenshots}{PD:T050}} \\
	Players may be able to take a screenshot of his current game that will be saved in the respective game's folder (\ref{FR:200}).
\end{description}