\subsection{Mandatory criteria}
\label{mandatory-criteria}
\subsubsection{Framework}
The framework provides rudimental functions that either all or the most \glspl{game} need. Such as creating new \glspl{graph} from specifications and visualizing and manipulating existing graphs.\par
These graphs consist of \glspl{edge} and \glspl{vertex} that themselves are equipped with parameters, such as style or weight. Parameters that are related to vertices are for example weight, length or direction.\par
Thus, a \gls{developer} using {\graphioli} does not need to implement a \gls{GUI} for his or her game.\par
Furthermore the framework lightens the developer's workload by providing access to reoccuring basic graph and \gls{path} finding algorithms.\par
{\graphioli} will offer interfaces for creating multiplayer modes so that two \glspl{player} are able to play against each other on the same computer sharing the same peripherals.\par
Games can be saved and reloaded by their player at a later time without losing \gls{level} stats.\par
In order to facilitate easy implementations, {\graphioli} comes with elaborate \Gls{javadoc} documentation from the viewpoint of the developer using the framework.\par

\subsubsection{Game-Explorer}
\todo{revise subsection}
The Game-Explorer is a graphical interface that builds the main contact point with the framework for a player. Without knowledge about the background of framework and implemented game logic, the player is still able to start and restart games from a list of available or previously saved games.\par
For that reason, the Game-Explorer can display information to every game, such as title, description and a static screenshot. It comes with a \emph{START} button \todo{No `buttons` in text} and a \emph{HELP} button for details about the rules and operability of the specific game.\par

\subsubsection{Games}
{\graphioli} comes with two implemented games.\par
The two-player strategy board game \textbf{TwixT} is one of most popular graph grames. Invented by board game designer \emph{Alexander Randolph} in the early 60s, it has since been awarded several prizes, including nomination for \emph{Spiel des Jahres} in 1979.\par
\todo{details for TwixT}
\todo{graph-coloring}\par