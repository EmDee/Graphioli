\subsection{Mandatory criteria}
\subsubsection{Framework}
The framework provides rudimental functions that either all or the most games need. Such as creating new graphs from specifications and visualizing and manipulating existing graphs.\par
These graphs consist of \glspl{edge} and \glspl{vertex} that themselves are equipped with parameters. Parameters for edges are at least style (background color, border color, background image, size), coordinates, weight, sorting, \gls{ID}. Parameters that are related to vertices are at least weight, length and direction.\par
Furthermore the framework lightens a \gls{developer}['s]\todo{Only use `developer` instead of `game developer`. Game developer needs to be deleted from glossary} workload by providing access to reoccuring basic graph algorithms. These include path recognition (depth-first search, breadth-first search) and Dijkstra.\par \todo{Remove details like depth-first search, breadth-first search and Dijkstra}
\graphioli will offer interfaces for creating multiplayer modes so that two \glspl{player} are able to play against each other on the same computer sharing the same peripherals.\par
Games can be saved and restored by their user at a later time without losing \gls{level} stats.\par
In order to facilitate easy implementations, \graphioli comes with elaborate \Gls{javadoc} documentation for the viewpoint of the developer using the framework.\par \todo{Make clear that developer doesn't need to implement a GUI}

\subsubsection{Game-Explorer}
The Game-Explorer is a graphical interface that builds the main contact point with the framework for a player. Without knowledge about the background of framework and implemented game logic, the player is still able to start and restart games from a list of available or previously saved games.\par
For that reason, the Game-Explorer can display information to every game, such as title, description and a static screenshot. It comes with a \emph{START} button \todo{No `buttons` in text} and a \emph{HELP} button for details about the rules and operability of the specific game.\par
The list of available games is dynamically created when the Game-Explorer starts. It reads a subfolder for legally specified games that are represented by a class that inherits from the frameworks game scaffold \todo{last sentence belongs to functional requirements}.

\subsubsection{Games}
\todo{Save, Load}