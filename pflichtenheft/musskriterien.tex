\subsection{Mandatory criteria}
\label{mandatory-criteria}
\subsubsection{Framework}
The framework provides rudimental functions that either all or the most \glspl{game} need. Such as creating new \glspl{graph} from specifications and visualizing and manipulating existing graphs.\par
These graphs consist of \glspl{edge} and \glspl{vertex} that themselves are equipped with parameters, such as style or weight. Parameters that are related to edges are for example weight, length or direction.\par
Thus, a \gls{developer} using {\graphioli} does not need to implement a \gls{GUI} for his or her game.\par
Furthermore the framework lightens the developer's workload by providing access to reoccuring basic graph and \gls{path} finding algorithms.\par
{\graphioli} will offer interfaces for creating multiplayer modes so that two \glspl{player} are able to play against each other on the same computer sharing the same peripherals.\par
Games can be saved and reloaded by their player at a later time without losing \gls{level} stats.\par
In order to facilitate easy implementations, {\graphioli} comes with elaborate \Gls{javadoc} documentation from the viewpoint of the developer using the framework.\par

\subsubsection{Game-Explorer}
\todo{revise subsection}
The \textbf{Game-Explorer} is a graphical interface that builds the main contact point with the framework for a \gls{player}. Without knowledge about the background of framework and implemented \gls{game} logic, the player is still able to start and restart games from a list of available or previously saved games, switch between single and multiplayer mode (if provided in the selected game) and so on.\par
For that reason, the Game-Explorer can display information to every game, such as title, description and a static screenshot.\par

\subsubsection{Games}
{\graphioli} comes with two already implemented games.\par
\textbf{\twixt}\par
The two-player strategy board game {\twixt} is one of most popular graph games. The two players have different colors that they use to color the vertices by turns in a 24x24 grid. Each player owns one of the two pairs of opposite border rows. Their contrary objective is to build a continuous \gls{path} of linked vertices connecting their border rows and try to interfere each other in accomplishing this objective.\par
Invented by board game designer \emph{Alexander Randolph} in the early 60s, {\twixt} has since been awarded several prizes, including nomination for \emph{Spiel des Jahres} in 1979\footnote{For more information see official page of \emph{Spiel des Jahres}: \url{http://www.spiel-des-jahres.com/cms/front_content.php?idcatart=474&id=370} (German)}.\par
\textbf{\graphcoloring}\par
The second game is a game variant of {\graphcoloring} in which the player (or players) tries to colorize the vertices of a graph in such a way that two \gls{adjacent} \glspl{vertex} never get the same color. The player is, depending on the level, restricted in the number of different colors that can be used and the number of \gls{undo} steps that he or she can perform.\par
In multiplayer mode, one players tries to block the other from achieving a fully colored graph.\par
A popular use case of the theory behind {\graphcoloring} is the smallest set of colors needed to colorize a map of neighboring countries (which is indeed a set containing a maximum of four colors, see \emph{\gls{four-color-theorem}}).\par