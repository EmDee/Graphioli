\subsection{Mandatory criteria}\label{REF:MANDATORY-CRITERIA}
\subsubsection{Framework}
\begin{itemize}

	\item The framework provides rudimental functions that most graph \glspl{game} need:
	\begin{itemize}
		\item Creation of new \glspl{graph},
		\item manipulation of existing graphs (removing or adding edges or vertices, changing parameters of edges and vertices) and 
		\item visualization of these graphs.
	\end{itemize}

	\item These graphs consist of \glspl{edge} and \glspl{vertex} that themselves are equipped with attributes. For vertices these are
	\begin{itemize}
		\item background icon,
		\item background color,
		\item stroke weight,
		\item stroke color,
		\item size,
		\item coordinates and
		\item a unique \gls{ID}.
	\end{itemize}
	Parameters that are related to edges are
	\begin{itemize}
		\item weight,
		\item color,
		\item width and
		\item direction.
	\end{itemize}
	These parameters can change during the course of the game (due to user action or game logic).

	\item The framework comes with default parameters for edges and vertices. The developer can additionally define own default sets of parameters. E.g. considering the subsequently explained \twixt, a developer would define a default set \emph{A} for edges and vertices of the first player and a default set \emph{B} for the second player.

	\item A \gls{developer} using \graphioli does not need to implement a \gls{GUI} for his or her game, because the framework provides all necessary utilities to listen for events, interact and display the graph and the game status.

	\item The graph is placed in a predefined grid system. The developer specifies the size, respectively the granularity of the grid.

	\item The framework reduces the developer's workload by providing access to the reoccuring basic graph and \gls{path} finding algorithms:
	\begin{itemize}
		\item \gls{BFS},
		\item an algorithm for checking if a path between given vertices exists and
		\item an algorithm for verifying if the drawing of the graph is \gls{planar}.
	\end{itemize}

	\item \graphioli will offer interfaces for creating multiplayer modes so that two or more \glspl{player} are able to play against each other on the same computer sharing the same peripherals.

	\item Games can be saved and reloaded by the player at a later time.

	\item In order to facilitate easy implementations, \graphioli comes with elaborate \Gls{javadoc} documentation for the developer using the framework.

\end{itemize}

\subsubsection{GUI}

\paragraph*{Game-Explorer}
The Game-Explorer is a graphical interface that builds the main contact point with the framework for a \gls{player}. Without knowledge about the background of the framework and the implemented \gls{game} logic, the player is able to start and restart games from a list of available or previously saved games and switch between single-player and multiplayer mode (if provided in the selected game).\par
For that reason, the Game-Explorer can display information to every game: the title, a description, a static screenshot and a detailed help file.
\paragraph*{Game window}
When a game is selected for playing, the game window opens. It shows a canvas containing the graph (or an empty area), the menu bar and the status bar.\par

\begin{samepage}
\subsubsection{Games}
\graphioli comes with two already implemented games.\par
\paragraph*\twixt
The two-player strategy board game \twixt is one of the most popular graph games. Both players take turns placing a vertex or an edge between vertices in their respective color in a square grid. The size of the grid can be chosen at the beginning of the game. Each player owns one of the two pairs of opposite border rows. Their contrary objective is to build a continuous \gls{path} of linked vertices connecting their border rows.\par
Two vertices can be linked by an edge, if they share a knight's move distance to each other. Since edges are not allowed to cross, a player can try to interfere the other in accomplishing this objective by blocking his or her way.\par
If both sides cannot achieve such a path, the game is a draw.\par
\end{samepage}
Invented by board game designer \emph{Alexander Randolph} in the early 60s, \twixt has since been awarded several prizes, including nomination for \emph{Spiel des Jahres} in 1979\footnote{For more information see official page of \emph{Spiel des Jahres}: \url{http://www.spiel-des-jahres.com/cms/front_content.php?idcatart=474&id=370} (in German)}.\par
\paragraph*\graphcoloring
The second game is a game variant of \graphcoloring in which the player (or players) tries to colorize the vertices of a graph in such a way that two \gls{adjacent} vertices never get the same color. The player is, depending on the level, restricted in the number of different colors that can be used.\par
In multiplayer mode, one players tries to block the other from achieving a fully colored graph. The player who performs the last valid move wins.\par