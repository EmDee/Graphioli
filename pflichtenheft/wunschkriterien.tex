\subsection{Facultative criteria}
\label{facultative-criteria}
Some functions we wanted to add, but may not have enough time to, will be introduced here. During the implementation there is still the possibility to add them.

\subsubsection{Framework}
Many \glspl{game} need additional (advanced) \gls{graph} \glspl{algorithm} (e.g. Dijksra, ...)	while the \gls{framework} still only supports few algorithms. \par
To be able to play multiplayer games alone the framework may provide an interface to implement an \gls{AI} and/or the possibility to play over network with friends. \par
While playing some time, single and even multiplayer games get monotonous when using the same graph all the time. There should be many different \glspl{level} to avoid this. A \gls{level-editor} could provide \glspl{developer} with the possibility to easily create several levels for their game. Another way is for the framework to generate random (\gls{planar}) graphs.
Within a game you may want to \gls{undo} (or redo) your last draw.
While it is possible to set or delete vertices the framework isn't able to pick and drag them.

\subsubsection{Game-Explorer}
The preferences of a game can be changed and saved. When restarting the game through the Game-Explorer it is possible to automatically load the preferences. \\
????

\subsubsection{Games}
other games we may want to implement \\
?????