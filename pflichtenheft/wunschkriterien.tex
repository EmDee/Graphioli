\subsection{Facultative criteria}\label{REF:FACULTATIVE-CRITERIA}
Some functions we wanted to add, but may not have enough time to, will be introduced here. During the implementation there is still the possibility to add them.

\subsubsection{Framework}
Many \glspl{game} need additional (advanced) \gls{graph} \glspl{algorithm} (e.g. Dijkstra, ...)	while the \gls{framework} still only supports few algorithms. We may extend the framework to increase its range of implementable graph games. \par
To extend the possibilities of multiplayer games {\graphioli} may provide an interface to implement an \gls{AI} and/or to play over network with friends. \par
While playing over the network there needs to be an option for players to communicate with each other. A \gls{chat} could provide such a communication. \par
Some games get tedious when using the same graph all the time. There should be many different \glspl{level} to avoid this. A \gls{level-editor} could provide \glspl{developer} with the possibility to easily create several levels for their game. Another way is for the framework to generate random graphs.
Within a game you may want to \gls{undo} (or \gls{redo}) your last draw.

\subsubsection{Game-Explorer}
The preferences of a game can be changed and saved. When restarting the game through the Game-Explorer it is possible to automatically load the preferences. \\
\todo{Add further descriptions.}

\subsubsection{Games}
\todo{Add further descriptions about other games we may want to implement.}