\subsection{Facultative criteria}\label{REF:FACULTATIVE-CRITERIA}
Some functions we wanted to add, but may not have enough time to, will be introduced here. During the implementation there is still the possibility to add them.

\subsubsection{Framework}
\begin{itemize}

	\item Many \glspl{game} need additional (advanced) \gls{graph} \glspl{algorithm}:
	\begin{itemize}
		\item Dijkstra\todo{Add further algorithms}.
	\end{itemize}

	\item To extend the possibilities of multiplayer games {\graphioli} may provide an interface to implement an \gls{AI}.

	\item {\graphioli} could feature network functionalities to support multiplayer games on more than one computer.

	\item While playing over the network there needs to be an option for players to communicate. A \gls{chat} could provide such a communication.

	\item Some games get tedious when using the same graph all the time. There should be many different \glspl{level} to avoid this. A \gls{level-editor} could provide \glspl{developer} with the possibility to easily create several levels for their game.

	\item The framework could generate random graphs instead of predefined ones.

	\item Within a game you may want to \gls{undo} (or \gls{redo}) your last draw.

	\item A timer could be provided to support games that reward a player's quickness\todo{'quickness' might be not the best choice.}.

	\item The framework could offer possibilities to define additional menu entries for the game window.

\end{itemize}

\subsubsection{Game-Explorer}
\begin{itemize}

	\item The preferences of a game can be changed and saved. When restarting the game through the Game-Explorer it is possible to automatically load the preferences.

	\item The player could add new games to the explorer by naming the path to a game file.

\end{itemize}

%\subsubsection{Games}
\todo{Add further descriptions about other games we may want to implement.}