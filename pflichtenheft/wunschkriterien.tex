\pagebreak
\subsection{Facultative criteria}\label{REF:FACULTATIVE-CRITERIA}
Some functions we wanted to add, but may not have enough time to, will be introduced here. During the implementation there is still the possibility to add them.

\subsubsection{Framework}
\begin{itemize}

	\item Many \glspl{game} need additional (advanced) \gls{graph} \glspl{algorithm}:
	\begin{itemize}
		\item Dijkstra Algorithm
		\item an algorithm to check if two graphs are isomorphic
	\end{itemize}

	\item The framework provides an interface to enable the developer to add further parameters for edges and vertices.
	\item To extend the possibilities of multiplayer games \graphioli may provide an interface to implement an \gls{AI}.

	\item \graphioli could feature network functionalities to support multiplayer games on more than one computer.

	\item While playing over the network there needs to be an option for players to communicate. An \gls{chat} could provide such a communication.

	\item A \gls{level-editor} could provide \glspl{developer} with the possibility to easily create several levels for their game.

	\item The framework could generate random graphs instead of predefined ones. The developer could define requirements for the graph:
	\begin{itemize}
		\item number and parameters of vertices and edges
		\item planarity of the graph
	\end{itemize}

	\item Within a game the players may want to \gls{undo} (or \gls{redo}) their last move.

	\item A timer could be provided to support games that reward a player's quickness.

	\item The framework could offer possibilities to define additional menu entries for the game window.

\end{itemize}

\subsubsection{Game-Explorer}
\begin{itemize}

	\item The player could add new games to the explorer from any location on his or her computer and from a central internet server.

\end{itemize}

\subsubsection{Games}
\begin{itemize}
	\item \twixt could be extended by adding the possibility to change the defined length of the edges and define all gridpoints as vertices from the start. Thus \textsc{Bridj-It} could be implemented.
	\item The \textsc{Shannon Switching Game} would be implemented by altering \twixt in another way: The players start with a graph with two special vertices. One player needs to connect these vertices with a path of colored edges and the other player tries to prevent that by removing uncolored edges.\footnote{For more information see \url{http://en.wikipedia.org/wiki/Shannon_switching_game}}
\end{itemize}