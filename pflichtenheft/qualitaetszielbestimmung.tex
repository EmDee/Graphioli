\section{Quality objective}

\begin{tabular}{lcccc}
\hline
\textbf{Criteria} & \textbf{very important} & \textbf{important} & \textbf{less important} & \textbf{least important} \\
\hline
Usability & X & & & \\
Portability & X & & & \\
Maintainability & X & & & \\
Reliability & & X & & \\
Efficiency & & & X & \\
Security & & & & X \\
\hline

\end{tabular}


\todo{Shouldn't we add consistency and stability? Please delete added definitions below, if not necessary.}
\begin{description}
	\item[Usability] Ease of use of the framework.
	\item[Portability] Capability of the framework to run in different environments and amount of effort it takes to adapt it to new environments.
	\item[Maintainability] The degree of effort required to correct software defects and to cope with user complaints\footnote{Fenton, Norman E.; Pfleeger, Shari L.: Software Metrics: A Rigorous and Practical Approach (1997)}.
	\item[Reliability] Ability of the framework to perform its functions without failure.
	\item[Efficiency] Ratio of the complexity of a task to the effort (time, memory space) necessary to achieve it.
	\item[Security] Level of protection against potential security breaches.
	\item[Consistency] Those attributes of the framework that provide uniform design and implementation techniques and notations.\footnote{McCall, J. A.; Richards, P. K., Walters, G. F.: Factors in Software Quality (1977)}
	\item[Stability] The benchmark of resistancy towards change of runtime environment, critical user interactions, workload and attacks.
\end{description}