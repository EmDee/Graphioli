% Non-functional requirements %

\subsection{Non-functional requirements}

%%%%%%%%%%% Framework %%%%%%%%%%%

\subsubsection{Framework}
\paragraph{Code}
\paragraph*{}
\begin{tabular}{{c}{l}}
    \hline
    \textbf{Functions} & \textbf{Description} \\ \hline
	\ref{NFR:F010} & \ref{NFR:FT010} \\
	\ref{NFR:F020} & \ref{NFR:FT020} \\
	\ref{NFR:F030} & \ref{NFR:FT030} \\
	\ref{NFR:F040} & \ref{NFR:FT040} \\ \hline
\end{tabular}

\vspace{.5cm}

\begin{description}
  	\item[\textlabel{/NFR010/}{NFR:F010}] \textbf{\textlabel{Simple usage}{NFR:FT010}}  \\
	\todo{Either without 'the' or 'Graphioli framework'}The {\graphioli} can be easily understood by the developer after a short period of induction. The induction should not take longer than one day for the developer to fully be able to work with the {\graphioli}.
	\item[\textlabel{/NFR020/}{NFR:F020}] \textbf{\textlabel{Algorithm efficiency}{NFR:FT020}}  \\ 
	The framework's graph algorithms have to be implemented in an efficient version.
	\item[\textlabel{/NFR030/}{NFR:F030}] \textbf{\textlabel{Development effort}{NFR:FT030}}  \\ 
	An experienced developer has to be able to implement a simple graph game within less than half an hour.
	\item[\textlabel{/NFR040/}{NFR:F040}] \textbf{\textlabel{Documentation in detail}{NFR:FT040}} \\
	The {\graphioli} has an simple to understand and easy accessible documentation. Each class, method will be described in detail with all attributes, parameters, and usage.
\end{description}

%%%%%%%%%%% GUI %%%%%%%%%

\paragraph{GUI}
\paragraph*{}
\begin{tabular}{{c}{l}}
    \hline
    \textbf{Functions} & \textbf{Description} \\ \hline
	\ref{NFR:GU010} & \ref{NFR:GUT010} \\ \hline
\end{tabular}

\vspace{.5cm}

\begin{description}
  	\item[\textlabel{/NFR050/}{NFR:GU010}] \textbf{\textlabel{UI Stability}{NFR:GUT010}}  \\
	The user interface has to remain functional, even if the implemented game is faulty.
\end{description}

%%%%%%%%%%% Explorer %%%%%%%%%

\paragraph{GUI}
\paragraph*{}
\begin{tabular}{{c}{l}}
    \hline
    \textbf{Functions} & \textbf{Description} \\ \hline
	\ref{NFR:GE010} & \ref{NFR:GET010} \\ \hline
\end{tabular}

\vspace{.5cm}

\begin{description}
	\item[\textlabel{/NFR060/}{NFR:GE010}] \textbf{\textlabel{Platzhalter}{NFR:GET010}}  \\
	Platzhalter. \todo{update this section}
\end{description}

%%%%%%%%%%% GAMES %%%%%%%%%%%

\subsubsection{Games}
\begin{tabular}{{c}{l}}
    \hline
    \textbf{Functions} & \textbf{Description} \\ \hline
	\ref{NFR:G010} & \ref{NFR:GT010} \\ \hline
\end{tabular}

\vspace{.5cm}

\begin{description}
	\item[\textlabel{/NFR070/}{NFR:G010}] \textbf{\textlabel{Termination}{NFR:GT010}}  \\
	Terminating a running game has to be possible under every circumstances. 
\end{description}