\subsection{Excluded criteria}
Das {\graphioli} framework bietet keine framework-bezogene GUI, die zum Beispiel das ``Zusammenklicken'' von Spielen erlauben würde. Die eingebundenen grafischen Oberflächen dienen nur zur Ausgabe des Spiels bzw. der Game-Explorer als Visualisierung der Übersicht vorhandener Spiele.\par
Es bietet außerdem keine Metasprache für die Implementierung von games. Der developer muss mit der Programmierung in Java vertraut sein und sich Kenntnisse über die Struktur der Bibliothek aneignen, als die dieses framework bereitgestellt wird.\par
Mit {\graphioli} können keine Spiele implementiert werden, die nicht auf Graphen basieren. Dabei nimmt das framework keinen Einfluss auf die Spiellogik, die Sache des Entwicklers ist. Es stellt nur die in \nameref{mandatory-criteria} spezifizierten und die eventuell aus \nameref{facultative-criteria} implementierten Funktionen zur Verfügung.\par
Das freie Zeichnen von edges und gebogene edges werden mit diesem framework nicht möglich sein.\par
\todo{translate}