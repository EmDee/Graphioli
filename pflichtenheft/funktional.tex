\subsection{Functional requirements}

%%%%%%%%%%% Framework %%%%%%%%%%%

\subsubsection{Framework}
\paragraph{Library}\label{FR:LIBRARY}
\paragraph*{}
\begin{tabular}{{c}{l}{c}}
	\hline
	\textbf{Function} & \textbf{Description} & \textbf{Optional} \\ \hline
	\ref{FR:L010} & \ref{FR:L010T} & {} \\
	\ref{FR:L020} & \ref{FR:L020T} & {} \\
	\ref{FR:L030} & \ref{FR:L030T} & {} \\
	\ref{FR:L040} & \ref{FR:L040T} & {} \\
	\ref{FR:L050} & \ref{FR:L050T} & {} \\
	\ref{FR:L060} & \ref{FR:L060T} & {} \\
	\ref{FR:L070} & \ref{FR:L070T} & {} \\
	\ref{FR:L080} & \ref{FR:L080T} & {} \\
	\ref{FR:L090} & \ref{FR:L090T} & {$\times$} \\
	\ref{FR:L100} & \ref{FR:L100T} & {} \\ \hline
\end{tabular}

\vspace{.5cm}

\begin{description}
	\item[\textlabel{/FR010/}{FR:L010}] \textbf{\textlabel{Set grid size}{FR:L010T}} \\
	Create the grid according to the size, set by the developer.
	\item[\textlabel{/FR020/}{FR:L020}] \textbf{\textlabel{Initialize graph}{FR:L020T}} \\
	Initialize empty graph structure.
	\item[\textlabel{/FR030/}{FR:L030}] \textbf{\textlabel{Add/Remove vertices/edges}{FR:L030T}} \\
	Add or remove \Glspl{vertex} to or from graph. \Glspl{edge} without one vertex at each end will be removed automatically. Furthermore the developer adds an edge connecting given vertices and removes edges between vertices.
	\item[\textlabel{/FR040/}{FR:L040}] \textbf{\textlabel{Interprete user input}{FR:L040T}} \\
	Interprete user input provided by \ref{FR:GU030}.
	\item[\textlabel{/FR050/}{FR:L050}] \textbf{\textlabel{Set attributes of vertices/edges}{FR:L050T}} \\
	Set the attributes of a vertex/edge. Vertices and edges have default attributes, if not set otherwise by the developer.
	\item[\textlabel{/FR060/}{FR:L060}] \textbf{\textlabel{Find path}{FR:L060T}} \\
	The framework implements algorithms to find a \gls{path} between given vertices and returns true, if path exists.
	\item[\textlabel{/FR070/}{FR:L070}] \textbf{\textlabel{Perform \gls{BFS}}{FR:L070T}} \\
	Return a list of reachable vertices within a set range from a given vertex.
	\item[\textlabel{/FR080/}{FR:L080}] \textbf{\textlabel{Check for intersecting edges}{FR:L080T}} \\
	Check if an edge between two given unconnected vertices exists, so that the resulting graph is still planarly drawn.
	\item[\textlabel{/FR090/}{FR:L090}] \textbf{\textlabel{Start/Stop timer}{FR:L090T}} \\
	Starts/Stops the timer.
	\item[\textlabel{/FR100/}{FR:L100}] \textbf{\textlabel{Set status}{FR:L100T}} \\
	Sets the status message with information about the current game.
\end{description}

%%%%%%%%%%% GUI %%%%%%%%%%%

\paragraph{GUI}\label{FR:GUI}
\paragraph*{}
\begin{tabular}{{c}{l}{c}}
	\hline
	\textbf{Function} & \textbf{Description} & \textbf{Optional} \\ \hline
	\ref{FR:GU010} & \ref{FR:GU010T} & {} \\
	\ref{FR:GU020} & \ref{FR:GU020T} & {} \\
	\ref{FR:GU030} & \ref{FR:GU030T} & {} \\
	\ref{FR:GU040} & \ref{FR:GU040T} & {$\times$} \\
	\ref{FR:GU050} & \ref{FR:GU050T} & {$\times$} \\ 
	\ref{FR:GU060} & \ref{FR:GU060T} & {$\times$}\\ \hline
\end{tabular}

\vspace{.5cm}

\begin{description}
	\item[\textlabel{/FR110/}{FR:GU010}] \textbf{\textlabel{Display grid}{FR:GU010T}} \\
	Display or hide grid layer.
	\item[\textlabel{/FR120/}{FR:GU020}] \textbf{\textlabel{Display graph}{FR:GU020T}} \\
	Display graph on top of grid system.
	\item[\textlabel{/FR130/}{FR:GU030}] \textbf{\textlabel{Capture user input}{FR:GU030T}} \\
	Capture mouse and keyboard input and forward it to the \nameref{FR:LIBRARY} for further processing. 
	\item[\textlabel{/FR140/}{FR:GU040}] \textbf{\textlabel{Display status}{FR:GU040T}} \\
	Display the message set in \ref{FR:L100} in the window bar. See Section \ref{REF:GUI_GAME} for further information.
	\item[\textlabel{/FR150/}{FR:GU050}] \textbf{\textlabel{Display timer}{FR:GU050T}} \\
	Display timer in the window bar, if activated by developer.
	\item[\textlabel{/FR160/}{FR:GU060}] \textbf{\textlabel{Add menu entry}{FR:GU060T}} \\
	Add additional menu entry to game options. See Section \ref{REF:GUI_GAME}.
\end{description}

%%%%%%%%%%% GAMES %%%%%%%%%%%

\begin{samepage}
\paragraph{Games}
\paragraph*{}
\begin{tabular}{{c}{l}{c}}
    \hline
    \textbf{Function} & \textbf{Description} & \textbf{Optional} \\ \hline
	\ref{FR:G010} & \ref{FR:G010T} & {} \\
	\ref{FR:G020} & \ref{FR:G020T} & {} \\
	\ref{FR:G030} & \ref{FR:G030T} & {$\times$} \\
	\ref{FR:G040} & \ref{FR:G040T} & {$\times$} \\
	\ref{FR:G050} & \ref{FR:G050T} & {$\times$} \\
	\ref{FR:G060} & \ref{FR:G060T} & {$\times$} \\
	\ref{FR:G070} & \ref{FR:G070T} & {$\times$} \\ \hline
\end{tabular}
\end{samepage}
\vspace{.5cm}

\begin{description}
	\item[\textlabel{/FR170/}{FR:G010}] \textbf{\textlabel{Save games}{FR:G010T}} \\
   	A started and running game can be saved by clicking the \emph{File} button in the menu bar and selecting \emph{Save}. The save function creates a \gls{savegame} file with the game's name and current system time as file name.
   	\item[\textlabel{/FR180/}{FR:G020}] \textbf{\textlabel{Load games}{FR:G020T}} \\
   	A saved game can be resumed by starting the game and selecting the \emph{File} button then \emph{Load}.
   	\item[\textlabel{/FR190/}{FR:G030}] \textbf{\textlabel{Pause games}{FR:G030T}} \\
   	While playing in single-player/multiplayer(local) mode, a game can be paused anytime. In multiplayer network mode a game can only be paused, if both parties agree to halt the game. A game can be paused by going into the \emph{Game} and selecting \emph{Pause}.
	\item[\textlabel{/FR200/}{FR:G040}] \textbf{\textlabel{Restart game}{FR:G040T}} \\
	A game can be restarted anytime during a single-player/multiplayer(local) mode game. Like \ref{FR:G030} a game can only be restarted in multiplayer network mode, if both parties agree upon it. Restarting can be accomplished by clicking \emph{Restart} in the menu bar under \emph{Game}.
	\item[\textlabel{/FR210/}{FR:G050}] \textbf{\textlabel{Undo move}{FR:G050T}} \\
	A move can be reverted like all the other game operations, by clicking the \emph{Game} entry in the menu bar and selecting \emph{Undo}. The \graphioli framework saves the game after every move. An undo move is basically a \emph{load} function as described in \ref{FR:G020}.
	\item[\textlabel{/FR220/}{FR:G060}] \textbf{\textlabel{Redo move}{FR:G060T}} \\
	Selecting \emph{Game} then \emph{Redo} loads the savegame before the last undo move.
	\item[\textlabel{/FR230/}{FR:G070}] \textbf{\textlabel{Take screenshots}{FR:G070T}} \\
	A player can take a screenshot of his current displayed game anytime during a running game. To take a screenshot, the player will have to go into the \emph{Game} menu entry and select \emph{Take Screen}. All screenshots will be saved in a subfolder within the game folder.
\end{description}

%%%%%%%%%%% Explorer %%%%%%%%%%%

\subsubsection{Game-Explorer}
\begin{tabular}{{c}{l}{c}}
	\hline
	\textbf{Function} & \textbf{Description} & \textbf{Optional} \\ \hline
	\ref{FR:GE010} & \ref{FR:GE010T} & {}  \\
	\ref{FR:GE020} & \ref{FR:GE020T} & {}  \\
	\ref{FR:GE030} & \ref{FR:GE030T} & {}  \\
	\ref{FR:GE040} & \ref{FR:GE040T} & {}  \\
	\ref{FR:GE050} & \ref{FR:GE050T} & {} \\ \hline
\end{tabular}

\vspace{.5cm}

\begin{description}
	\item[\textlabel{/FR240/}{FR:GE010}] \textbf{\textlabel{Scan game folder}{FR:GE010T}} \\
	Scan game folder for games and generate a file containing a listing of the scanned games.
   	\item[\textlabel{/FR250/}{FR:GE020}] \textbf{\textlabel{Show list of games}{FR:GE020T}} \\
	Using the file generated with \ref{FR:GE010} and show them in the \emph{Game-Explorer}.
  	\item[\textlabel{/FR260/}{FR:GE030}] \textbf{\textlabel{Start games}{FR:GE030T}} \\
	Once a game is selected in the menu, the player starts a game by pressing the \emph{Start} button. See Section \ref{REF:GAME-EXPLORER}.
  	\item[\textlabel{/FR270/}{FR:GE040}] \textbf{\textlabel{Display help page}{FR:GE040T}} \\
	A tutorial or instruction file for each game (provided by the developer), can be viewed by clicking the \emph{Help} button, which opens up a new window with the instruction/tutorial.
	\item[\textlabel{/FR280/}{FR:GE050}] \textbf{\textlabel{Display description}{FR:GE050T}} \\
	When a game is selected, the \emph{Game-Explorer} will display an image and a description of the game, which both are provided by the developer.
\end{description}